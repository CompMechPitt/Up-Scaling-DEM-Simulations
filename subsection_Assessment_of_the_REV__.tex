\subsection{Assessment of the REV Size}

Homogenizing DEM simulations necessitates the existence and determination of the REV for that medium. Generally, the REV of a given domain can fundamentally be described as the smallest subdomain that is sufficiently large as to be statistically representative of the entire domain \citep{Kanit_2003, Gitman_2007}. This qualitative definition is insufficient to rigorously define an REV quantitatively, for the REV characterization is subjective with respect to what constitutes something to be “statistically representative”. As such, the assessment of the REV can be a contentious issue, fraught with ambiguity.

One can conceptualize an REV to be “statistically representative” in two primarily different ways \citep{Drugan_1996}. The classically cited means for characterizing an REV suggests that the micro-scale heterogeneities (e.g. fractures, voids, grains, etc.) should be statistically representative within the REV such that the REV should contain a sufficiently large sample of these heterogeneities. This characterization of the REV is potentially problematic when attempting to quantify the REV, such that the descriptions of these heterogeneities tend to be nominally qualitative, and at best, quasi-quantitative.

The alternative means of conceptualizing “statistically representative”, and arguably the more pragmatic way when considering numerical modeling, proposes that the constitutive response of the REV should be statistically representative of the domain. Unlike the other characterization, the constitutive response of a subdomain is quantifiable through resultant model properties and parameters. This interpretation the REV is used in this investigation and has been widely used in other numerical studies due to its quantifiability \citep{Kanit_2003, Gitman_2005, Gusev_1997, M_ller_2010}.
