%Flow Rule, same as concrete damage. should bring to main d-p material
%description:

%\begin{equation}
%G=\sqrt{\left[\epsilon\bar{\sigma}_{0}\tan\left(\psi\right)\right]^{2}+q^{2}}-p\tan\left(\psi\right)\label{eqn:druc5}
%\end{equation}


%default value of eccentricity: maybe

%\begin{equation}
%\epsilon=\label{eqn:druc5-1}
%\end{equation}


Here, isotropic hardening is assumed, such that the friction angle remains constant with respect to changing stresses. The hardening function in this formulation is considered tensorially as a function of the effective stress and the equivalent plastic strain which can be written as:

\begin{equation}
h_{ij}\left(\bar{\sigma}_{ij},\bar{\epsilon}^{pl}\right)=\frac{1}{\sigma_c\left(\bar{\boldsymbol{\epsilon}}^{pl}\right)}\sigma_{ij}
\label{eqn:druc6}
\end{equation}


%where:

%\begin{equation}
%d'=\sqrt{l_{0}^{2}+\sigma_{c}^{2}}-\frac{\sigma_{c}}{3}\tan\left(\beta\right)\label{eqn:druc6-1}
%\end{equation}



The damage initiation criterion for this material model was based off of the Johnson-Cook model of damage initiation \cite{Johnson_1985}. For this model, a ductile damage formulation was assumed. At large confining pressures at large scale, the damage behaviour of rock can be considered to be behaving in a ductile capacity. More specifically, the Johnson-Cook model assumes the equivalent plastic strain when damage is initiated is a function of triaxiality, $\eta$, equivalent plastic strain rate, $\dot{\bar{\epsilon}}^{pl}$, and temperature, $\hat{T}$:

\begin{equation}
\bar{\epsilon}_{f}^{pl}\left(\eta,\dot{\bar{\epsilon}}^{pl},\hat{T}\right)=\left[D_{1}+D_{2}e^{D_{3}\eta}\right]\left[1+D_{4}\ln\left(\frac{\dot{\bar{\epsilon}}^{pl}}{\dot{\bar{\epsilon}}}\right)\right]\left[1+D_{5}\hat{T}\right]\label{eqn:druc7}
\end{equation}

Because the expected loading for these geomaterials does not have a high degree of thermal variance and the loading tends to be very slow, isothermal conditions and negligible strain rate effects are assumed to yield the following form of the Johnson-Cook damage initiation model:

\begin{equation}
\bar{\epsilon}_{f}^{pl}\left(\eta\right)=\left[D_{1}+D_{2}e^{D_{3}\eta}\right]\label{eqn:druc8}
\end{equation}

After the material has experienced yield and material damage has occurred, the stress-strain relationship becomes strongly mesh-dependant due to strain localization due to the energy dissipation decreasing as the mesh is refined. As such, \citet{Hillerborg_1976} proposed a stress-displacement response based on fracture energy after damage initiation. The effective plastic strain rate is related to the plastic displacement rate, $\dot{\bar{u}}^{pl}$, by the characteristic element length, $L$, as follows:

\begin{equation}
\dot{\bar{u}}^{pl}=L\dot{\bar{\epsilon}}^{pl}\label{eqn:druc9}
\end{equation}

The damage evolution model used in this constitutive model assumes that damage is a progressively linear degradation of the material stiffness in compression.  Here, a mesh independent measure of the plastic displacement, $\bar{u}^{pl}$ is used to to govern the evolution of damage to make the formulation more universal:


Assuming a linear form, the plastic displacement when the material is completely damaged, $(\bar{u}^{pl}_f)$, can be specified, and the damage evolution can then be written as:

\begin{equation}
\dot{D}=\frac{L\dot{\bar{\epsilon}}^{pl}}{\dot{\bar{u}}_{f}^{pl}}=\frac{\dot{\bar{u}}^{pl}}{\dot{\bar{u}}_{f}^{pl}}\label{eqn:druc9-1}
\end{equation}
