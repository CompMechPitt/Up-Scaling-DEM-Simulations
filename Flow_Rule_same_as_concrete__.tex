%Flow Rule, same as concrete damage. should bring to main d-p material
%description:

%\begin{equation}
%G=\sqrt{\left[\epsilon\bar{\sigma}_{0}\tan\left(\psi\right)\right]^{2}+q^{2}}-p\tan\left(\psi\right)\label{eqn:druc5}
%\end{equation}


%default value of eccentricity: maybe

%\begin{equation}
%\epsilon=\label{eqn:druc5-1}
%\end{equation}


The hardening rule can be written as

\begin{equation}
h\left(\boldsymbol{\sigma},\bar{\sigma}\right)=\frac{\boldsymbol{sigma}}{\bar{\sigma}\left(\bar{\boldsymbol{\epsilon}}^{pl}\right)}
\label{eqn:druc6}
\end{equation}


%where:

%\begin{equation}
%d'=\sqrt{l_{0}^{2}+\sigma_{c}^{2}}-\frac{\sigma_{c}}{3}\tan\left(\beta\right)\label{eqn:druc6-1}
%\end{equation}


isotropic hardening is assumed, treating friction angle constant wrt
stress

{*}Damage initiation

- assume ductile damage. at large confining pressures at large scale,
the behaviour of rock is more ductile than brittle.

- onset of damage due to `` nucleation, growth, and coalescnce of
voids''. Model assumes PEEQ when damage is initiated is a function
of triaxiality, eta.

assume the form of the Johnson-Cook Model:

\begin{equation}
\bar{\epsilon}_{f}^{pl}\left(\eta,\dot{\bar{\epsilon}}^{pl},\hat{T}\right)=\left[D_{1}+D_{2}e^{D_{3}\eta}\right]\left[1+D_{4}\ln\left(\frac{\dot{\bar{\epsilon}}^{pl}}{\dot{\bar{\epsilon}}}\right)\right]\left[1+D_{5}\hat{T}\right]\label{eqn:druc7}
\end{equation}


assuming isothermal conditions and neglecting strain rate effects
because of reasons:

\begin{equation}
\bar{\epsilon}_{f}^{pl}\left(\eta\right)=\left[D_{1}+D_{2}e^{D_{3}\eta}\right]\label{eqn:druc8}
\end{equation}


{*}Damage evolution:

- assumes damage is a progressive degredation of the material stiffness

- assume isotropic damage

- uses mesh independant measure of plastic dispalcement to drive evolution
of damage

- damage manifests itself in two forms: softenin of the yeild stress
and degredation of the elastisity.

Damage evolution equation based on effective plastic displacement:

\begin{equation}
\dot{\bar{u}}^{pl}=L\dot{\bar{\epsilon}}^{pl}\label{eqn:druc9}
\end{equation}


asuming a linear form, the plastic displacement at at complete damage
($D=1$) can be specified, and the damage evolution can be written
as:

\begin{equation}
\dot{d}=\frac{L\dot{\bar{\epsilon}}^{pl}}{\dot{\bar{u}}_{f}^{pl}}=\frac{\dot{\bar{u}}^{pl}}{\dot{\bar{u}}_{f}^{pl}}\label{eqn:druc9-1}
\end{equation}
