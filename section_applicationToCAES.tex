\section{Comparison to DNS - Application to Slope Stability Analysis}

To validate the up-scaling methodology used, a simple 2-D slope problem is presented and loaded from the top until failure using both DEM and the up-scaled CDM model. Here, the resultant stress distribution are compared just as failure occurs. 

In the DEM model, failure can be assessed based on the unbalanced forces in the model. Since the joints in the model have a stiffness and cohesion, when the slope fails, the explicit quasi-static solution becomes dynamic because of a sudden release of elastic energy and the inability of the applied damping to suppress it all. At this point, the total unbalanced forces in the model increase and the slope can be said to have failed. 

For the CDM model, failure can be assessed based on non-convergence of the model when run as an implicit static simulation, which does not converge when the slope fails. The load step in which the CDM model fails to converge because the slope fails dynamically is considered the point of failure.
