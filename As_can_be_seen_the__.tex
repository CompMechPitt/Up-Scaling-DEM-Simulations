As can be seen, the continuum approximation of the stress fields in the slope shows a very good match to the smoothed DEM stress fields. More importantly however is the load at failure for the two models were very close. The DEM slope simulation failed at $11.2 MPa$, while the CDM slope simulation failed at $11.5 MPa$, resulting in an error ($~3\%$) that is not only negligable in the context of geological uncertainty but acceptable due to the computational savings. This agreement of the two models both in terms of the stress distribution and the failure load shows a high degree of success of the up-scaling framework. 

An additional comparison of the surface deflection where the load was applied is presented in Figure \ref{fig:surfacedeflection}. In this figure, one can see that the general behaviour of the two models is similar, with a downward displacement occurring where the load is applied, upwards displacement towards the slope on the left and negligible displacement towards the right model boundary. 