\section{Plasticity Based Damage Mechanics}


\subsection{....damage}

Damage mechanics allow for the 

The constitutive stress-strain relationship including a scalar damage
parameter, $\mathbf{D}$ can be written as follows: 
\begin{equation}
\boldsymbol{\sigma}=(1-\mathbf{D})\mathbf{E}:\boldsymbol{\epsilon^{el}}\label{eqn:const3}
\end{equation}


For simplicity, the damaged elastic stiffness is described as the
reduced stiffness due to the damage: 
\begin{equation}
\mathbf{E^{d}}=(1-\mathbf{D})\mathbf{E}\label{eqn:const4}
\end{equation}



\subsection{....plasticity}

Plasticity based damage mechanics models attempt to incorporate the
theories of plasticy and damage machanics into a unified approach
to the damage evolution and constituive relationships (Zhang and Cai,
2010). These 

Damage plasticity model based on Strain can be decomposed into elastic
and plastic components: 
\begin{equation}
\boldsymbol{\epsilon}=\boldsymbol{\epsilon}^{el}+\boldsymbol{\epsilon}^{pl}\label{eqn:const1}
\end{equation}


Substituting \ref{eqn:const1} and \ref{eqn:const4} into \ref{eqn:const3}
results in the following: 
\begin{equation}
\boldsymbol{\sigma}=\mathbf{E^{d}}:(\boldsymbol{\epsilon}-\boldsymbol{\epsilon}^{pl})\label{eqn:const5}
\end{equation}


Using the \textquotedbl{}usual notions of CDM\textquotedbl{} (find
reference), the effective stress. $\boldsymbol{\bar{\sigma}}$, can
be defined as: 
\begin{equation}
\boldsymbol{\bar{\sigma}}=\mathbf{E}:(\boldsymbol{\epsilon}-\boldsymbol{\epsilon}^{pl})\label{eqn:const6}
\end{equation}


Such that the cauchy stress tensor can be related to the effective
stress tensor as follows: 
\begin{equation}
\boldsymbol{\sigma}=(1-\mathbf{D})\boldsymbol{\bar{\sigma}}\label{eqn:const7}
\end{equation}


The nature of the damage evolution is assumed to be a function of
the effective stress and the equivalent plastic strain, $\boldsymbol{\bar{\epsilon}^{pl}}$:
\begin{equation}
\mathbf{D}=\mathbf{D}(\boldsymbol{\bar{\sigma}},\boldsymbol{\bar{\epsilon}^{pl}})\label{eqn:const8}
\end{equation}


The evolution of the equivalent plastic strains are described by the
time derivative of the equivalent plastic strain. This rate of equivalent
plastic strain which can be considered to be related to the time derivative
of the plastic strain through a hardening rule, $\mathbf{h}$ such
that: 
\begin{equation}
\boldsymbol{\dot{\bar{\epsilon}}^{pl}}=\mathbf{h}(\boldsymbol{\bar{\sigma}},\boldsymbol{\bar{\epsilon}^{pl}})\bullet\boldsymbol{\dot{\epsilon}}\label{eqn:const10}
\end{equation}


The flow rule can be written in terms of the flow potential function,
$G(\boldsymbol{\bar{\sigma}})$, and a plastic mulitplier $\dot{\lambda}$:
\begin{equation}
\boldsymbol{\dot{\epsilon}}=\dot{\lambda}\dfrac{\partial G(\boldsymbol{\bar{\sigma}})}{\partial\boldsymbol{\bar{\sigma}}}\label{eqn:const11}
\end{equation}


Non-associated plasticity is used, which required the solution of
non-symetric equations.


\subsection{....brittle vs ductile damage...}


\subsection{....misc for now....}

Taking the time derivative gives the decomposition of the strain rate,
$\boldsymbol{\dot{\epsilon}}$ : 
\begin{equation}
\boldsymbol{\dot{\epsilon}}=\boldsymbol{\dot{\epsilon}}^{el}+\boldsymbol{\dot{\epsilon}}^{pl}\label{eqn:const2}
\end{equation}
