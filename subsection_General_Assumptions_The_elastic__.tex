\subsection{General Assumptions}
The elastic response of the material can be fully characterized with just Young's modulus, $E$, and Poisson's ratio $\nu$. Assuming plane strain conditions, the elastic response can be written in terms of these two parameters:
\begin{equation}
\boldsymbol{\sigma}_{ij}=\frac{1+v}{E}\left\{\boldsymbol{\epsilon}_{ij}+\frac{\nu}{1-2\nu}\boldsymbol{\epsilon}_{kk}\delta_{ij}\right\}
\label{eqn:const8a}
\end{equation}

Furthermore, the plasticity models used here are assumed to comprise of three key components: The flow rule, the hardening rule and the yield function. The flow rule, in general, describes the amount of plastic deformation that the material should exhibit given an applied stress. The flow rule in these models is assumed to be of the following incremental form:

\begin{equation}
d\boldsymbol{\epsilon_p}=d\lambda \frac{\partial G\left(\bar{\boldsymbol{\sigma}}\right)}{\partial \boldsymbol{\sigma}}
\label{eqn:const8b}
\end{equation}

Where $\epsilon_p$ is the plastic strain, $\lambda$ is a hardening parameter, and $G\left(\bar{\boldsymbol{\sigma}}\right)$ is the flow potential function. In both models presented here, the flow potential function is taken from the Drucker-Prager model:

\begin{equation}
G\left(\boldsymbol{\bar{\sigma}}\right)=\sqrt{\left(\varepsilon\sigma_{t0}\tan\psi\right)^{2}-\bar{q}^{2}}-\bar{p}\tan\psi\label{eqn:const11}
\end{equation}

\begin{equation}
G=\sqrt{\left[\epsilon\bar{\sigma}_{0}\tan\left(\psi\right)\right]^{2}+q^{2}}-p\tan\left(\psi\right)\label{eqn:druc5}
\end{equation}



%In rock mechanics, the material models assume that the plastic strain increment and the and the normal to the yield surface have the same direction.
etc..

The nature of the damage evolution is assumed to be a function of the effective stress, $\boldsymbol{\bar{\sigma}}$, and the equivalent plastic strain, $\boldsymbol{\bar{\epsilon}^{pl}}$:
\begin{equation}
D=D(\boldsymbol{\bar{\sigma}},\boldsymbol{\bar{\epsilon}^{pl}})\label{eqn:const8}
\end{equation}



Here, the effective stress,  can be described as a stress that the system would be experiencing without any stiffness degradation or damage. This stress can be related to the actual Cauchy stress through the damage variable: 
\begin{equation}
\boldsymbol{\sigma}=(1-D)\boldsymbol{\bar{\sigma}}\label{eqn:const7}
\end{equation}

etc..

From ..., $q$ and $p$ represent two stress invariants, the mises
equivalent stress and the equivalent pressure stress(hydrostatic stress):

\begin{equation}
p=-\frac{1}{3}tr\left(\boldsymbol{\sigma}\right)\label{eqn:druc3}
\end{equation}


\begin{equation}
q=\sqrt{\frac{3}{2}}\left(\mathbf{S}:\mathbf{S}\right)\label{eqn:druc4}
\end{equation}


where $S$ is the stress deviator:

\begin{equation}
\mathbf{S}=\boldsymbol{\sigma}+p\mathbf{I}\label{eqn:druc4-1}
\end{equation}


and I is the second order identity tensor. 


Also, algebraically maximum eigenvalue of effective stress, $\hat{\bar{\sigma}}$


etc...

For both of the material models presented here, the plasticity models are similar. 