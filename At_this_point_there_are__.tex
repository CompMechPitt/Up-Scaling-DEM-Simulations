At this point, there are some parameter constraints for the damage evolution that need to be considered for numerical stability. In this model, The damage curves are specified in terms of inelastic strain and cracking strain which need to be converted into plastic strain for the analysis. The inelastic and cracking strains represent the same strain component but refer to compression and tension respectively. This inelastic/cracking strain can be considered as the theoretical plastic strain given that the material is in an undamaged state. The conversion from inelastic/cracking strain to plastic strain is a function of the damaged state at every increment so the :

\begin{equation}
\label{eqn:param11}
\begin{array}{c}
\bar{\epsilon}_{c}^{pl}=\bar{\epsilon}^{in}-\frac{D_{c}}{1-D_{c}}\frac{\sigma_{c}^{iy}}{E}\\
\bar{\epsilon}_{t}^{pl}=\bar{\epsilon}^{ck}-\frac{D_{t}}{1-D_{t}}\frac{\sigma_{t}^{iy}}{E}
\end{array}
\end{equation}


The numerical issues with this formulation arise due to the fact that it is very possible for the converted plastic strain to not be monotonically increasing with respect to the tensile damage. By having a damage evolution curve with a sufficiently steep slope, such that the second term in equation \ref{eqn:param11} increases faster than the first term, it becomes mathematically possible to have decreasing and/or negative plastic strains in the damage evolution definition. As such, the following conditions is applied to constrain the damage evolution to always yield monotonically increasing plastic strains as damage increases:

\begin{equation}
\label{eqn:param6-1}
\begin{array}{c}
\frac{d\bar{\epsilon}_{c}^{pl}}{dD_{c}}>0\\
\frac{d\bar{\epsilon}_{t}^{pl}}{dD_{t}}>0
\end{array}
\end{equation}


For the compressive damage evolution, we can substitute equation \ref{eqn:param3} into \ref{eqn:param11} to get the following expression of plastic strain as a function of the compressive damage rate parameter:

\begin{equation}
\bar{\epsilon}_{c}^{pl}=\bar{\epsilon}^{in}-\frac{\bar{\epsilon}^{in}m}{1-\bar{\epsilon}^{in}m}\frac{\sigma_{c}^{iy}}{E}\label{eqn:param12}
\end{equation}

Combining \ref{eqn:param12} and \ref{eqn:param6-1} yields the following expression governing the stability limit for the compressive damage rate parameter:

\begin{equation}
m<\frac{\sigma_{c}^{iy}+2E\bar{\epsilon}^{in}-\sqrt{\sigma_{c}^{iy}\left(\sigma_{c}^{iy}+4E\bar{\epsilon}^{in}\right)}}{2E\left(\bar{\epsilon}^{in}\right)^{2}}\label{eqn:param13}
\end{equation}

However, since this upper bound for $m$ is functional on several of the parameterization parameters, which are not constant, the upper bound varies depending on the other input parameters. Because of this, a compressive damage scaling factor, $d_c$, is introduced. In addition, the upper bound for $m$ is dependent on the inelastic strain which is not constant throughout the model. Since only one value of $m$ can be specified for a given simulation, the chosen value of $m$ should be the smallest value over the range of the expected inelastic strain experienced. As can be seen from equation \ref{eqn:param13}, as $\bar{\epsilon}^{in} \rightarrow \infty$, $m\rightarrow0$ such that for very large inelastic strains the conversion to plastic strain becomes very unstable. Thus, the compressive damage rate parameter can be written as:

\begin{equation}
m=d_{c}\min_{\bar{\epsilon}^{in}}\left\{\frac{\sigma_{c}^{iy}+2E\bar{\epsilon}^{in}-\sqrt{\sigma_{c}^{iy}\left(\sigma_{c}^{iy}+4E\bar{\epsilon}^{in}\right)}}{2E\left(\bar{\epsilon}^{in}\right)^{2}}\right\}
\label{eqn:param14}
\end{equation}


Where the compressive damage scaling factor has the following limits:

\begin{equation}
0<d_{c}<1\label{eqn:param15}
\end{equation}


The tensile damage evolution curve has the same numerical constraints when converting form cracking strain to plastic strain. Substituting equation \ref{eqn:param4} into equation \ref{eqn:param11} yields the following expression for the plastic strain:

\begin{equation}
\bar{\epsilon}_{t}^{pl}=\bar{\epsilon}^{ck}-\left[\left(1+\bar{\epsilon}^{ck}\right)^{n}-1\right]\frac{\sigma_{t}^{iy}}{E}\label{eqn:param5-1}
\end{equation}


Solving for $n$ with equations \ref{eqn:param6-1} and \ref{eqn:param5-1} yeilds the following inequality governing the upper bound of the tensile damage rate parameter:

\begin{equation}
n<\frac{W\left(\frac{E\left(\bar{\epsilon}^{ck}+1\right)}{\sigma_{t}^{iy}}\ln\left(\bar{\epsilon}^{ck}+1\right)\right)}{\ln\left(\bar{\epsilon}^{ck}+1\right)}\label{eqn:param7}
\end{equation}

Where $W\left(x\right)$ is the Lambert W function defined implicitly as \cite{Corless_1996}:

\begin{equation}
x=W\left(x\right)e^{W(x)}\label{eqn:param8}
\end{equation}

Using the same methodology as was used to derive equation \ref{eqn:param14}, the tensile damage scaling factor can be written as:


\begin{equation}
n=d_{t}\min_{\bar{\epsilon}^{ck}}\left\{\frac{W\left(\frac{E\left(\bar{\epsilon}^{ck}+1\right)}{\sigma_{t}^{iy}}\ln\left(\bar{\epsilon}^{ck}+1\right)\right)}{\ln\left(\bar{\epsilon}^{ck}+1\right)}\right\}
\label{eqn:param9}
\end{equation}


where the tensile damage scaling factor has the following limits:

\begin{equation}
0<d_{t}<1\label{eqn:param10}
\end{equation}

The parameterization of the concrete damaged-plasticity model yielded a total of 9 required parameters to characterize the model:2 elastic parameters, 5 plastic parameters and 2 damage parameters. 

\begin{table}[]
\centering
\caption{My caption}
\label{my-label}
\begin{tabular}{cccc}
\hline
\multicolumn{2}{c}{Parameter Type}                         & Name                               & Symbol             \\ \hline
\multicolumn{2}{c}{\multirow{2}{*}{Elastic}}               & Young's Modulus                    & $E$                \\
\multicolumn{2}{c}{}                                       & Poisson's Ratio                    & $\nu$              \\ \cline{1-2}
\multirow{7}{*}{Plastic} & \multirow{2}{*}{Flow Rule}      & Dilation Angle                     & $\psi$             \\
                         &                                 & Flow Eccentricity                  & $\varepsilon$      \\ \cline{2-2}
                         & \multirow{2}{*}{Yield Function} & Second Stress Invariant Ratio      & K                  \\
                         &                                 & Initial Equibiaxial Stress Ratio   & $frac{fb_0}{fc_o}$ \\ \cline{2-2}
                         & \multirow{3}{*}{Hardening Rule} & Initial Compressive Yield Strength & $\sigma_c^{iy}$    \\
                         &                                 & Peak Compressive Yield Strength    & $\sigma_c^{p}$     \\
                         &                                 & Strain at Peak Compressive Yield   & $\epsilon_c^{pp}$  \\ \cline{1-2}
\multirow{2}{*}{Damage}  & Compressive                     & Compressive Damage Scaling Factor  & $d_c$              \\ \cline{2-2}
                         & Tensile                         & Tensile Damage Scaling Factor      & $d_t$             
\end{tabular}
\end{table}