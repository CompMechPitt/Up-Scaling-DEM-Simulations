\section{Parameter Estimation}
Parameter estimation involves a process of obtaining a parameter set $\boldsymbol{\chi}$ of a CDM model that minimizes the difference between $\boldsymbol{\sigma}^M$-$\boldsymbol{\epsilon}^M$ and $\left<\boldsymbol{\sigma}\right>$-$\left<\boldsymbol{\epsilon}\right>$ for all load paths. Herein, the parameter estimation was conducted using calibration algorithms, a subset of optimization which attempts to minimize a least-squares objective function \cite{matott_ostrich:_2008}. Optimization algorithms are often described as either deterministic (local search) or heuristic (global search). Deterministic optimization algorithms primarily focus on searching for the optima within the local parameter space by iteratively converging towards a solution. Heuristic optimization algorithms explore the entire parameter space approximately and provide an estimate of the global optima. Heuristic techniques are useful for highly non-linear problems, where there are numerous local optima within the prescribed parameter space. When searching the global parameter-space deterministically becomes too computationally demanding, heuristic methods are used, at the cost of completeness and accuracy. A compromise between speed and accuracy can be obtain by strategically using both types of algorithms.

A combination of two optimization algorithms is used to assess the optimal parameter set. An initial heuristic algorithm is applied to search for the approximate global optima, followed by a deterministic algorithm as a local refinement of the optimal parameter set. Particle Swarm Optimization (PSO) is used for the global heuristic search, whereas the Levenberg-Marquardt Algorithm (LMA) is used for the local deterministic search. 

%The PSO algorithm was developed by  as a byproduct of modeling the cooperative-competitive nature of social behaviour in birds as they flocked searching for food. The PSO algorithm, in a conceptual sense, consists of a series of 'particles' (birds) which 'swarm' through the entire parameter space (sky) searching for the global optima (food) using a combination of individual 'particle' knowledge and global 'swarm' (flock) knowledge.

%The Levenburg-Marquardt Algorithm (LMA) was proposed by \citet{marquardt_algorithm_1963} which builds off of the work of \citet{levenberg_method_1944}. This calibration algorithm combines a quasi-Newton approach with a conjugate gradient technique in order to efficiently minimize non-linear least-squares problems. 

The parameter estimation works by iteratively running a single element CDM model, subject to boundary conditions provided by the homogenized DEM simulations, with successive parameter sets that intelligently adapt in order to converge to the DEM data. 
