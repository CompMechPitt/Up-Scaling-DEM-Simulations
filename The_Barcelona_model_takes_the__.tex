The Barcelona model takes the following form in order to capture the key hardening and softening behaviour of the material, and can be written in terms of the initial compressive yield strength $\sigma_c^{iy}$, and two material parameters, $\alpha$ and $\beta$:

\begin{equation}
%\sigma_{c}\left(\bar{\epsilon}^{in}\right)=\frac{\sigma_{c}^{iy}-\sigma_{c}^{p}}{\left(\epsilon_{c}^{pp}\right)^{2}}\left(\bar{\epsilon}^{in}-\epsilon_{c}^{pp}\right)^{2}+\sigma_{c}^{p}
\sigma_c=\sigma_c^{iy}\left [ \left ( 1+\alpha \right ) e^{-\beta\bar{\epsilon}^{pl}}-\alpha e^{-2\beta\bar{\epsilon}^{pl}}  \right ]
\label{eqn:param2-1}
\end{equation}

It was found to be useful when applying bounding limits for the parameters to manipulate \ref{eqn:param2-1} to allow for the governing parameters to have a physical meaning. As such, $\alpha$ and $\beta$ can be rewritten in terms of the peak compressive yield strength, $\sigma_{c}^{p}$, and the plastic strain at the peak compressive yield strength, $\epsilon_c^{pp}$:

\begin{equation}
\beta=\frac{\ln\left[\frac{2\alpha}{1+\alpha} \right ]}{\epsilon_c^{pp}}
\label{eqn:param2-2}
\end{equation}

\begin{equation}
\alpha =\frac{2\sigma_c^{p}-\sigma_c^{iy}+2\sqrt{-\sigma_c^p\left(\sigma_c^{iy}-\sigma_c^p \right )}}{\sigma_c^{iy}}
\label{eqn:param2-3}
\end{equation}

Here, the compressive yield stress, $\sigma_{c}$, is written as function
of the inelastic strain, $\bar{\epsilon}^{in}$, and three additional
parameters. The three parameters are the initial compressive yield
stress ($\sigma_{c}^{iy}$), the peak compressive yield stress ($\sigma_{c}^{p}$),
and the plastic strain at the peak compressive yield stress ($\sigma_{c}^{iy}$).
The physical significance of each of these parameters can be seen
in Fig \ref{fig:conccomp}, where they define the y-intercept and the peak of the curve.

The tensile hardening rule has a fundamentally different behavior
than the compressive hardening rule, and was therefore approximated
using an exponential function (Fig \ref{fig:conctens}). The exponential function required
only two parameters to characterize the curve completely. The first
parameter was the initial tensile yield stress, $\sigma_{t}^{iy}$,
which defines the y-intercept of the curve, while the second parameter
was the tensile yield stress decay parameter,$\lambda$. These parameters
describe the relationship between the tensile yield stress, $\sigma_{t}$,
and the cracking strain, $\bar{\epsilon}^{ck}$, and has the form:

\begin{equation}
\sigma_{t}\left(\bar{\epsilon}^{ck}\right)=\sigma_{t}^{iy}e^{\lambda\bar{\epsilon}^{ck}}\label{eqn:param2}
\end{equation}


