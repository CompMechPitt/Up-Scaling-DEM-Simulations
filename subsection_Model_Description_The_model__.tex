\subsection{Model Description}

The model geometry for the slope stability problem was constructed at a height of $50m$ and a depth of $80m$ with a $30m$ high slope with a grade of $300\%$ as can be seen in Figure \ref{fig:slopeGeom}. This geometry provided enough space for the failure mechanisms to occur without influence from the boundaries. For this slope, the lateral boundaries had fixed displacements in the x direction, and the bottom boundary had fixed displacements in the y direction. The slope and top boundaries were free. A uniformly distributed load was applied over a 5m section representing a quasi-point load at a rate of $2.5 MPa/s$. 
