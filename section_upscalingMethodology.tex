\section{Overview of the Up-Scaling Methodology}
The goal of the up-scaling methodology is to identify the parameters of a continuum constitutive model (macroscale model) that best emulates the average response of the DEM REV (microscale model). Let the displacement, strain and stress of the DEM REV (microscale) model be denoted by $\mathbf{u}^m$, $\boldsymbol{\epsilon}^m$, and $\boldsymbol{\sigma}^m$, respectively. Let the homogenized (averaged) strain and stress of the DEM REV model be denoted by  $\left<\boldsymbol{\epsilon}\right>$ and $\left<\boldsymbol{\sigma}\right>$, respectively.  Finally, let the strain and stress from the continuum (Macroscale) constitutive model be denoted as $\boldsymbol{\epsilon}^M$ and $\boldsymbol{\sigma}^M$, respectively. The rate of macroscale stress, $\dot{\boldsymbol{\sigma}}^M=\dot{\boldsymbol{\sigma}}^M\left(\dot{\boldsymbol{\epsilon}}^M, \boldsymbol{\chi},\mathbf{h}\right)$, is defined in terms of the rate of macroscale strain, $\dot{\boldsymbol{\epsilon}}^M$, a set of material parameters $\boldsymbol{\chi}$ and a set of internal history variables $\mathbf{h}$.

The up-scaling methodology has five steps: 
\begin{enumerate}
    \item Identify the DEM REV for the NFR.
	\item Exercise the DEM REV using multiple load paths. Store $\mathbf{u}^m$, $\boldsymbol{\epsilon}^m$, and $\boldsymbol{\sigma}^m$ for each load path.
	\item Apply homogenization algorithms to the microscale results ($\mathbf{u}^m$, $\boldsymbol{\epsilon}^m$, and $\boldsymbol{\sigma}^m$) to determine the average stress-strain response of the REV, i.e., $\left<\boldsymbol{\sigma}\right>$-$\left<\boldsymbol{\epsilon}\right>$, for each load path.
	\item Identify a continuum constitutive model, $\dot{\boldsymbol{\sigma}}^M=\dot{\boldsymbol{\sigma}}^M\left(\dot{\boldsymbol{\epsilon}}^M, \boldsymbol{\chi},\mathbf{h}\right)$, that captures the salient features of NFR mechanics.
	\item Run parameter estimation algorithms to identify the parameters, $\boldsymbol{\chi}$, that minimize the difference between $\left<\boldsymbol{\sigma}\right>$-$\left<\boldsymbol{\epsilon}\right>$ and $\boldsymbol{\sigma}^M$-$\boldsymbol{\epsilon}^M$ over all load paths.
\end{enumerate}

Once an optimal parameter set, $\boldsymbol{\chi}$, for the desired model, $\dot{\boldsymbol{\sigma}}^M=\dot{\boldsymbol{\sigma}}^M\left(\dot{\boldsymbol{\epsilon}}^M, \boldsymbol{\chi},\mathbf{h}\right)$, has been identified, the newly established constitutive model can be used in Finite Element Method (FEM) models or with any other numerical or analytical simulation.

\subsection{Software Implementation}
The up-scaling framework that is presented here consists of four main software components (Figure \ref{fig:workflow}): a DEM simulator, a homogenization module, a FEM simulator, and a parameter estimation module. In procedural order, the first software component involved is a DEM simulation package, which is used to directly model the NFR. The DEM software accepts as inputs the geometry of the DFN, the material properties of the rock and the natural fractures, and the load paths. The DEM REV is exercised for different load-paths in a way that is akin to conducting multiple triaxial tests on physical specimens to characterize the full range of material behaviour. The DEM software outputs the microscale displacement, $\mathbf{u}^m$, and stress-strain, $\boldsymbol{\sigma}^m$-$\boldsymbol{\epsilon}^m$, responses for each load path.  This microscale data is subsequently fed into the homogenization module to compute the average stress-strain response, $\left<\boldsymbol{\sigma}\right>$-$\left<\boldsymbol{\epsilon}\right>$,  for each load path. Next, the homogenized stress-strain data, $\left<\boldsymbol{\sigma}\right>$-$\left<\boldsymbol{\epsilon}\right>$, is used by the parameter estimation software as observation data (i.e., laboratory/field data). The parameter estimation module iteratively executes a constitutive model, $\dot{\boldsymbol{\sigma}}^M=\dot{\boldsymbol{\sigma}}^M\left(\dot{\boldsymbol{\epsilon}}^M, \boldsymbol{\chi},\mathbf{h}\right)$, embedded in the FEM simulator for each load path using different parameter sets, $\boldsymbol{\chi}^i$, while attempting to minimize the error between the homogenized microscale, $\left<\boldsymbol{\sigma}\right>$-$\left<\boldsymbol{\epsilon}\right>$, and macroscale, $\boldsymbol{\sigma}^M$-$\boldsymbol{\epsilon}^M$,  stress-strain curves. Eventually, the algorithm converges to a near-optimal parameter set, $\boldsymbol{\chi}$, that can be viewed to be the best estimate of the NFR responses by the given continuum model. In our implementation, UDEC\textsuperscript{TM} was used as the DEM simulator and ABAQUS\textsuperscript{TM} was used as the FEM simulator. In ABAQUS single element simulations were performed where a prescribed strain-history is prescribed through displacement boundary conditions for a given set of material parameters and the stress is obtained as the output. There is nothing particularly special about the DEM or FEM simulators chosen and each can easily be envisions to be replaced in the future to overcome any inherent limitations.  More over, a FEM simulator is not actually needed, since its inclusion into the implementation of the framework is simply to gain access the constitutive models within.  The FEM simulator could easily be replaced by a FDM simulator or simply by a material subroutine.  OSTRICH, a model-independent optimization package written by \citet{matott_ostrich:_2016}, is used for the parameter estimation module. The homogenization module was written in-house in Python. Python was also used to interface and drive the various components of the up-scaling framework. 




