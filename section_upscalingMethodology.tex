\section{Overview of the Up-Scaling Methodology}
The goal of the up-scaling methodology is to identify the parameters of a continuum constitutive model that best emulates the response of the DEM REV. The up-scaling methodology has five steps: 
\begin{enumerate}
    \item Identify the DEM RVE for the NFR.
	\item Exercise the  DEM RVE using multiple load paths.
	\item Apply homogenization algorithms to DEM (microscale) results to determine the average (macroscale) stress-strain response of the RVE.
	\item Identify a continuum constitutive model suitable to capture the salient features of NFR mechanics.
	\item Run parameter estimation algorithms to minimize the difference between the continuum and homogenized DEM responses.
\end{enumerate}

\subsection*{Identify DEM RVE}
Identifying the appropriate size of an RVE for a given NFR is a challenging task, which is often based on trial and error. Clearly, the RVE should be sufficiently large such that the homogenized stress-strain response for the DEM RVE has converged. In our experience, a safe RVE size for a NFR with non-uniform fracture spacing is about 10 times the average spacing between fractures. The role of RVE size is further investigated below.

\subsection*{Exercise DEM RVE}
The DEM RVE is exercised for different load-paths in a way that is akin  to conducting multiple laboratory tests on physical specimens to characterize the full range of material behaviour. Specifically, the mechanical response of NFR depends the degree of confinement and so it is necessary to perform multiple virtual triaxial tests a different levels of confinement. 

\subsection*{Apply homogenization algorithms}
Output from the DEM simulation include the nodal displacements of the deformable blocks between the natural fractures and therefore the normal and tangential crack opening displacements. From the displacements, the microscale stress in the deformable blocks can be determined.  Using these quantities, the average (macroscale) stress-strain behaviour of the DEM RVE is determined. The homogenization algorithm is discussed in detail in Section XXX.

\subsection*{continuum constitutive model)
In dealing with NFR, the behaviour of the rock mass is largely controlled by the reactivation and failure of existing natural fractures. As fractures grow, the integrity and stiffness of the rock mass decreases. From this perspective, a continuum description of a NFR can be well approximated using Continuum Damage Mechanics (CDM). CDM models capture the salient features of progressive failure and stiffness degradation without the need for an explicit representation of the fracture network. Here we adopted a Ducker-Prager model characterized by 11 parameters-2 elastic, 6 plastic, and 3 damage.

\subsection*{run parameter estimation algorithms}
In this step, the 11 parameters of the CDM which lead to the CDM that is best able to approximate the average (macroscale) stress-strain behaviour of the RVE for all of the load-paths explored  are determined using an iterative optimization algorithm. The optimization algorithm is discussed further in Section YYY.

\subsection(software implementation)
The upscaling framework that is presented here consists of four main software components (Figure \ref{fig:workflow}). In procedural order, the first software component involved is a DEM simulation package. The DEM software is used to generate a dataset of DEM displacements and stresses, which is subsequently fed into the homogenization program to produce the average (macroscale) stress and strain tensors over the whole range of strains and for each load path. At this point, the homogenized stress-strain data is used by the parameter estimation software as observation data (i.e., laboratory text data). The parameter estimation program, using optimization algorithms, iteratively executes the CDM model for each load path using different parameter sets, while attempting a least-squares minimization of the error between the homogenized and CDM stress-strain curves. Eventually, the algorithm converges to a near-optimal parameter set that can be viewed to be the best estimate of the NFR responses by the given continuum model.

Once the optimal parameter set for the CDM model has been identified, the newly established constitutive model can be used in Finite Element Method (FEM) models or with any other numerical or analytical model.


