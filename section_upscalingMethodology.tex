\section{Up-Scaling Methodology}
The upscaling methodology that is developed in this paper aims to emulate the constitutive response of DEM simulations within a continuum framework. This emulation is achieved though homogenization of the DEM simulations in order to get an average stress-strain response of the system. A parameterized continuum model is subsequently fit to this DEM response using a series of optimization algorithms to ultimately estimate an optimal set of parameters for the continuum model.

%In general, standard geomechanical material models tend to be elastoplastic in nature. 

In dealing with NFR, Continuum Damage Mechanics (CDM) model. In order for this method to be effective, the CDM has to be parameterized by identifying key parameters in the constitutive relationships that govern the behavioral response. The general upscaling methodology presented here can be summarized in three steps:
\begin{enumerate}
	\item Run REV DEM simulations of NFR under various loading conditions.
	\item Apply homogenization algorithms to DEM results to obtain stress-strain curves.
	\item Iteratively run a parameterized CDM model within a parameter estimation algorithm to minimize the difference between the CDM and DEM responses.
\end{enumerate}
Once the optimal parameter set for the continuum material model is identified, the newly established constitutive model can be used in a Finite Element Method (FEM) code to simulate the response of NFR at the reservoir scale.

Add here some of the implementation mechanics