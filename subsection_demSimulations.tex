\subsection{DEM Simulations}

The DEM models used in this investigation consisted of a pseudo-random isotropic fracture network which was defined by a Voronoi tessellation. The average block size was specified to be 0.5m using 20 iterations of LLoyd's method \citep{Lloyd_1982} in order to achieve a even size distribution. A 10m x 10m domain was determined to be sufficiently large to represent the rock mass behaviour as an REV. 

A Mohr-Coulomb Plasticity model was used as the constitutive model to describe the plastic behaviour of the intact  (deformable blocks) and the joint (natural fracture) behaviour was governed by a Coulomb area slip model. The parameters for the rock and joints were chosen to be representative of a fractured granitic rock mass. The rock and joint properties used for the DEM simulations are summarized in Table \ref{tab:demProp}. Due to the material properties of the DEM model, the joints are relatively weak compared to the blocks and so the blocks behave mostly elastically. 

\begin{table}[!htbp]
\centering
\caption{Rock and joint properties for DEM Simulations}
\label{tab:demProp}
\begin{tabular}{@{}lll@{}}
\toprule
Property Type          & Property         & Value        \\ \midrule
\multirow{7}{*}{Rock}  & Young's Modulus  & $65 GPa$     \\
                       & Poisson's Ratio  & $0.2$        \\
                       & Density          & $2.7 g/cm^3$ \\
                       & Friction Angle   & $51^{\circ}$ \\
                       & DilationAngle    & $0^{\circ}$  \\
                       & Cohesion         & $55.1 MPa$   \\
                       & Tensile Strength & $11.7 MPa$   \\ \cmidrule(r){1-1}
\multirow{6}{*}{Joint} & Friction Angle   & $32^{\circ}$ \\
                       & Dilation Angle   & $5^{\circ}$  \\
                       & Cohesion         & $100 kPa$    \\
                       & Tensile Strength & $0 kPa$      \\
                       & Normal Stiffness & $10 GPa/m$   \\
                       & Shear Stiffness  & $1 GPa/m$    \\ \bottomrule
\end{tabular}
\end{table}


The blocks were meshed with linear three-node triangular plane strain finite difference elements with an average side length of 0.5m. This discretization yielded 5-10 zones within each block. A rounding length of 10\% of the average block edge length (0.05m) was applied to the blocks in order to prevent numerical instabilities in the contact algorithm. Quasi-static analysis is obtained through dynamic relaxation, in which the dynamic equations are integrated in time using velocity-proportional viscous damping and mass scaling. State data of the model was collected at 50 evenly spaced intervals. 

The quasi-static loading of the DEM simulations was conducted to imitate triaxial laboratory tests. Here, a constant confining stress was applied on the lateral boundaries of the DEM model. The model is compressed by applying vertical  displacements to the top boundary while fixing the bottom boundary.The model was compressed to an axial strain of $5\%$ in the vertical direction for four confining horizontal stresses: $0.5MPa$, $1MPa$, $2MPa$, and $4MPa$. These load paths capture key physical phenomena including the pressure dependent yield of the NFR, hardening, and the dependence of damage initiation on the triaxiality. 

---- Remove? 
Second, triaxial compressive strength test at a strain rate of $0.05\%/s$ for $50s$ followed by a tensile strain rate of $0.05\%/s$ for $50s$ to observe the stiffness degradation and elastic recovery response of the DEM simulations during unloading. Here, even though the extended Drucker-Prager model is formulated for monotonic loading, the cyclic loading capacity of this model was investigated. The idea with these simulations was to strain the model past the yield point in order to investigate the post-yield behavior, but not strain the model to failure such that it loses all of its strength. ---- Remove? 

