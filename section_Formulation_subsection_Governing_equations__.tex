\section{Formulation}

\subsection{Governing equations of the coarse-scale (continuum) model}
Consider the dynamic equilibrium of a naturally fractured rock mass $\Omega$. Let $\Gamma$ denote the boundary of $\Omega$ and let $\Gamma$ be divided into mutually exclusive sets $\Gamma_u$ and $\Gamma_t$.   The body contains a set of natural fractures denoted by $\Gamma_{cr}$ and is subjected to a body force $\mathbf{g}$.  Let material points in the the undeformed and the deformed configuration be denoted by $\mathbf{X}$ and $\mathbf{x}$, respectively. Let $\mathbf{u}\left(\mathbf{X}, t\right)=\mathbf{x}\left(\mathbf{X}, t\right)-\mathbf{X}$ denote the displacement of material point $\mathbf{x}$ at time $t$.    Equilibirum of $\Omega$ is governed by
\begin{equation}
\label{eqn:equil}
\rho_s \ddot{\mathbf{u}} =\nabla \cdot \boldsymbol{\sigma} +\mathbf{g},\:\forall \mathbf{x}\in\Omega, t\geq0,
\end{equation}
in which $\ddot{\mathbf{u}}=\ddot{\mathbf{u}}\left(\mathbf{x}, t\right)$ denotes the second partial derivative of the displacement field and $\rho_s$ is the density of the rock mass. 





\subsection{Stress Homogenization}

\begin{equation}
\label{eqn:stressav}
\langle \boldsymbol{\sigma} \rangle_\Omega = \frac{1}{|A|} \int_\Omega \boldsymbol{\sigma} dA
\end{equation}

\begin{equation}
\label{eqn:stresssplit}
\langle \boldsymbol{\sigma} \rangle_\Omega = \frac{1}{\vert A \vert} \lbrack \int_{\Omega_{R}} \boldsymbol{\sigma} dA_R + \left \int_{\Omega_{C}} \boldsymbol{\sigma} dA_C \rbrack
\end{equation}

