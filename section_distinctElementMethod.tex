\section{Distinct/Discrete Element Method}
Discontinuous systems are characterized by the existence of discontinuities that separate discrete domains within the system. In order to effectively model a discontinuous system, it becomes necessary to represent two distinct types of mechanical behaviour: the behaviour of the discontinuities and the behaviour of the solid material.

There exists a set of methods, referred to as Discrete Element Methods, which provide the capacity to explicitly represent the behaviour of multiple intersecting discontinuities. The methods allow for the modelling of finite displacements and rotations of discrete bodies, including contact detachment as well as automatic detection of new contacts. Within the set of Discrete Element Methods, \citet{CUNDALL_1992} describes four subsets: Modal Methods, Discontinuous Deformation Analysis Methods, Momentum Exchange Methods, and Distinct Element Methods (DEM).

With DEM methods, the discontinuous system is represented as an assembly of deformable blocks such that the interfaces between the blocks represent the discontinuities. With respect to NFR, the blocks can be used to represent the intact rock while the discontinuities represent the joints in the rock mass. 

Consider an arbitrary deformable domain, $\Omega$, with a boundary, $\Gamma$, that is subdivided by discontinuities into $i$ number of subdomains, each denoted by $\Omega_i$ (Figure \ref{fig:DEM}). Let $\Gamma_{ij}=\Gamma_{ji}$ represent the boundary between $\Omega_i$ and $\Omega_j$. The motion of these subdomains (Herein referred to as elements) is governed by the conservation of momentum which relates the divergence of the stress field within the element, $\nabla\boldsymbol{\sigma}^m$, to the element acceleration, $\ddot{\mathbf{u}}^m$, and density, $\rho$:

\begin{equation}
\rho \ddot{\mathbf{u}}^m=\nabla\cdot\boldsymbol{\sigma}^m
\label{eqn:cauchy}
\end{equation}

