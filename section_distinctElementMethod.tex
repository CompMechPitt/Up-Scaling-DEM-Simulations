\section{Distinct Element Method}
Discontinuous systems are characterized by the existence of discontinuities that separate discrete domains within the system. In order to effectively model a discontinuous system, it becomes necessary to represent two distinct types of mechanical behaviour: the behaviour of the discontinuities and the behaviour of the solid material.

There exists a set of methods, referred to as Discrete Element Methods, which provide the capacity to explicitly represent the behaviour of multiple intersecting discontinuities. The methods allow for the modelling of finite displacements and rotations of discrete bodies, including contact detachment as well as automatic detection of new contacts. Within the set of Discrete Element Methods, \citet{CUNDALL_1992} describes four subsets: Modal Methods, Discontinuous Deformation Analysis Methods, Momentum Exchange Methods, and Distinct Element Methods (DEM).

DEM distinguishes itself from the other Discrete Element Methods by using an explicit time-marching scheme to directly solve the equations of motion. In addition the discrete domains in DEM can be considered rigid or deformable, and the contacts betweeen the domains are treated as deformable.  

%DEM is a numerical method targeted at modeling discontinuous systems. 

With DEM methods, the discontinuous system is represented as an assembly of discrete blocks such that the interfaces between the blocks represent the discontinuities. With respect to NFR, the blocks can be used to represent the intact rock while the discontinuities represent the joints in the rock mass.  

In this paper, ITASCA\textsuperscript{TM}'s implementation of a 2-dimensional DEM code, UDEC\textsuperscript{TM} was used for all the DEM simulations. With UDEC\textsuperscript{TM}, the elements of the discretized DEM mesh are referred to as blocks. If the blocks are considered to be deformable, they are discretized into a Finite Difference Method (FDM) mesh. The elements associated with this mesh are referred to as zones. The nodes that define the FDM zones are referred to as grid points, and the grid points that exit on the edge of the blocks are a special kind of gripoint refered to as corners. When two blocks interact with eachother, a contact is defined which governs the block interaction behvaiour. The space inbetween theses two contacts is refered to as a domain, which is relavent for fluid flow and pore pressure calculations. Figure \ref{fig:dem} shows schematically the relation of all six of these key components. 