\subsection{General Assumptions}
The elastic response of the material can be fully characterized with just Young's modulus and Poisson's ratio $\nu$. Assuming plane strain conditions, the elastic stiffness tensor can be written as:
\begin{equation}
\epsilon_{ij}=\frac{1+v}{E}\left\{\epsilon_{ij}+\frac{\nu}{1-2\nu}\epsilon_{kk}\delta_{ij}\right\}
\label{eqn:const8a}
\end{equation}
etc..

The nature of the damage evolution is assumed to be a function of the effective stress, $\boldsymbol{\bar{\sigma}}$, and the equivalent plastic strain, $\boldsymbol{\bar{\epsilon}^{pl}}$:
\begin{equation}
D=D(\boldsymbol{\bar{\sigma}},\boldsymbol{\bar{\epsilon}^{pl}})\label{eqn:const8}
\end{equation}



Here, the effective stress,  can be described as a stress that the system would be experiencing without any stiffness degradation or damage. This stress can be related to the actual Cauchy stress through the damage variable: 
\begin{equation}
\boldsymbol{\sigma}=(1-D)\boldsymbol{\bar{\sigma}}\label{eqn:const7}
\end{equation}

etc..

From ..., $q$ and $p$ represent two stress invariants, the mises
equivalent stress and the equivalent pressure stress(hydrostatic stress):

\begin{equation}
p=-\frac{1}{3}tr\left(\boldsymbol{\sigma}\right)\label{eqn:druc3}
\end{equation}


\begin{equation}
q=\sqrt{\frac{3}{2}}\left(\mathbf{S}:\mathbf{S}\right)\label{eqn:druc4}
\end{equation}


where $S$ is the stress deviator:

\begin{equation}
\mathbf{S}=\boldsymbol{\sigma}+p\mathbf{I}\label{eqn:druc4-1}
\end{equation}


and I is the second order identity tensor. 


Also, algebraically maximum eigenvalue of effective stress, $\hat{\bar{\sigma}}$


etc...

For both of the material models presented here, the plasticity models are similar. 