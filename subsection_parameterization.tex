\subsection{ModelParameterization}
parameterization of the material models is essential to effective upscaling of the system. parameterization needs to be conducted in such a way as to minimize the number of parameters, while still yielding a sufficiently accurate response. in addition, it is ideal, but not necessary for the parameters to have physical meaning in order to specify realistic bounds with less difficulty.

In order to accurately assess the optimal parameter set, the parameter estimation of the CDM material model was conducted in two steps. The first step consisted of a uniaxial tension test to estimate the elastic and tensile plastic properties of the material. The test direction reverses to compression before complete yield of the material in order to assess the tensile damage parameter. 

Parameterization of plastic CDM models involves the parameterization of all six key components of plastic CDM models:

For both the concrete damaged plasticity model and the Drucker-Prager model, the elastic behaviour is able to be parameterized with just Young's elastic modulus and Poisson's ratio. These two parameters are able to fully characterize the elastic stiffness tensor.

Parameterization of flow potential from equation \ref{eqn:const11} can be done directly using the eccentricity and the dilation angle.

Parameterization of the compressive and tensile hardening functions:

The parameterization of the yeild functions, and damage behaviour for the two models differ in approach and assumptions and are described below:





In addition, the damage evolution was parameterized using only the plastic displacement at failure parameter. The complete list of all 13 parameters can be found in Table \ref{tab:paramdruc}

\begin{table}[]
\centering
\caption{Parameters for Drucker-Prager Plasticity Model with Ductile Damage}
\label{tab:paramdruc}
\begin{tabular}{@{}cccc@{}}
\toprule
\multicolumn{2}{c}{Parameter Type}                         & Name                               & Symbol                                \\ \midrule
\multicolumn{2}{c}{\multirow{2}{*}{Elastic}}               & Young's Modulus                    & $E$                                   \\
\multicolumn{2}{c}{}                                       & Poisson's Ratio                    & $\nu$                                 \\ \cmidrule(r){1-2}
\multirow{7}{*}{Plastic} & \multirow{4}{*}{Flow Rule/ }      & Dilation Angle                     & $\psi$                                \\
                         & Yield Function                   & Flow Eccentricity                  & $\varepsilon$                         \\ 
                         &                                  & Friction Angle                     & $\beta$                               \\
                         &                                 & Initial Tensile Strength           & $p_0^t$                               \\ \cmidrule(lr){2-2}
                         & \multirow{3}{*}{Hardening Rule} & Initial Compressive Yield Strength & $\sigma_c^{iy}$                       \\
                         &                                 & Peak Compressive Yield Strength    & $\sigma_c^{p}$                        \\
                         &                                 & Strain at Peak Compressive Yield   & $\epsilon_c^{pp}$                     \\ \cmidrule(r){1-2}
\multirow{4}{*}{Damage}  & \multirow{3}{*}{Initiation}     & Yeild Strain at -0.5 Triaxiality      & $\bar{\epsilon}^{pl}_{f_{-0.5}}$   \\
                         &                                 & Yeild Strain at -0.75 Triaxiality    & $\bar{\epsilon}^{pl}_{f_{-0.75}}$ \\ \cmidrule(lr){2-2}
                         & Evolution                       & Plastic Displacement at Failure    & $\bar{u}^{pl}_f$                      \\ \bottomrule
\end{tabular}
\end{table}