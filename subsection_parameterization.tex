\subsection{Physically Meaningful Model Parameterization}
To accelerate the process of finding a near-optimal set of parameters, it is important to limit the search space of the parameterization algorithm. This is especially important when the number of parameters is large. We have found that it is beneficial for the parameters to have physical meaning in order to specify realistic bounds.

The elastic behaviour is written in defined in terms of Young's Modulus, $E$, and Poisson's Ratio, $\nu$. Bounds of these quantities of straightforward to define and so $E$, and $\nu$ are used as parameters. The yield function and flow potential function are also parameterized in terms of the friction angle, dilation angle, and the stress ratio $K$. Again, defining bounds on these quantities is relatively straightforward to define.

The hardening function (\ref{eqn:param2-1}) is given in terms of two empirical coefficients $\alpha$ and $\beta$ and the initial compressive yield stress $\sigma_c^{iy}$.  While it is possible to set bounds on $\sigma_c^{iy}$, it is less straightforward to set bounds for $\alpha$ and $\beta$ since they don't have obvious physical meaning. The hardening function for the Barcelona model is shown in Figure \ref{fig:barcelona}. The coefficients $\alpha$ and $\beta$ can be rewritten in terms of the peak compressive yield strength, $\sigma_{c}^{p}$, the plastic strain at the peak compressive yield strength, $\epsilon_c^{p}$, and the initial compressive yield stress $\sigma_c^{iy}$:
