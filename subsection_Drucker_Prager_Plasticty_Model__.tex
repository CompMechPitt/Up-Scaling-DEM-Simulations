\subsection{Drucker-Prager Plasticty Model With Ductile Damage}
the Drucker-Prager plasticity model was developed by by \citet{drucker_implications_1950} for modeling frictional materials like granular soils and rock. These materials tend  to exhibit pressure dependent yielding (as confining
pressure increases, so does the strength of the rock) which was incorporated as a pressure dependent yield criterion in the Drucker-Prager plasticity model. 

Specifically, the Drucker-Prager material model is formulated and used for materials of which the compressive yield strength is much greater than the tensile yield strength such as one would find in soils and rocks. Howewver, one drawback with this material model is that it is intended to simulate material response under essentially monotonic loading which limits the potential of modeling cyclic loading tonight.

In addition, the Drucker-Prager model is suitable for using in conjunction with progressive damage and failure models. In this formulation, the Johnson-Cook Damage model is used to model the ductile damage of the rock mass \cite{johnson_fracture_1985}. At a sufficiently large scale, the damage behaviour of NFR can be thought of as behaving in a ductile capacity. 

The elastic response of the material can be fully characterized with just Young's modulus, $E$, and Poisson's ratio $\nu$. Assuming plane strain conditions

Furthermore, the plasticity models used here are assumed to comprise of three key components: The yield function, the flow rule and the hardening rule. The yield function indicates whether or not the material has experienced yield given a particular stress state. The yield function varies between the two models but can be written in general as a function of effective stress and equivalent plastic strain:

\begin{equation}
F\left(\bar{\sigma}_{ij}, \bar{\epsilon}^{pl}\right)=\frac{1}{2}q\left(\bar{\sigma}_{ij}\right)\left [ 1+\frac{1}{K}-\left ( 1-\frac{1}{K} \right )\left ( \frac{r\left(\bar{\sigma}_{ij}\right)}{q\left(\bar{\sigma}_{ij}\right)} \right )^3 \right ]-p\left(\bar{\sigma}_{ij}\right)\tan\beta
\label{eqn:const8c}
\end{equation}

The flow potential function is taken from the Drucker-Prager model:

\begin{equation}
G\left(\bar{\sigma}_{ij}\right)=\sqrt{\left[\varepsilon\sigma^{iy}\tan\psi\right]^{2}+q\left(\bar{\sigma}_{ij}\right)^{2}}-p\left(\bar{\sigma}_{ij}\right)\tan\psi\label{eqn:const11}
\end{equation}

Where $\psi$ is the dilation angle of the material, $\sigma^{iy}$ is the initial yield stress, $\varepsilon$ is the eccentricity of the flow potential, while $p$ and $q$ stress invariants represent the mises equivalent stress and the equivalent pressure stress (hydrostatic stress) and are defined as follows:

\begin{equation}
%-\frac{1}{3}tr\left(\boldsymbol{\sigma}\right)
p\left(\bar{\sigma}_{ij}\right)=\frac{1}{3}\bar{\sigma}_kk
\label{eqn:druc3}
\end{equation}

\begin{equation}
%\sqrt{\frac{3}{2}}\left(\mathbf{S}:\mathbf{S}\right)
q\left(\bar{\sigma}_{ij}\right)=\sqrt{\frac{3}{2}S_{ij}S_{ij}}\label{eqn:druc4}
\end{equation}

Where $S_{ij}$ is known as the stress deviator with $\delta_{ij}$ being the Kronecker Delta:

\begin{equation}
%\mathbf{S}=\boldsymbol{\sigma}+p\mathbf{I}\label{eqn:druc4-1}
S_{ij} = \sigma_ij + p\left(\bar{\sigma}_{ij}\right)\delta_{ij}
\end{equation}

In addition to the yield function and the flow rule, the hardening rule is prescribed to govern the increase/decrease in yield stress as the plastic strain increases. More specifically, the hardening function, $h$, in these models is used to relate the equivalent plastic strain, $\bar{\epsilon}^{pl}$,  to the plastic strain in rate form: 

\begin{equation}
    \dot{\bar{\epsilon}}^{pl} 
    = 
    h_{ij}
    \left(
        \bar{\sigma}_{ij}, \bar{\epsilon}^{pl}
    \right)
    \dot{\epsilon}^{pl}_{ij}
\label{eqn:const8d}
\end{equation}

%In rock mechanics, the material models assume that the plastic strain increment and the and the normal to the yield surface have the same direction.
etc..

For the damage models, the damage initiation criteria and evolution equations are different for each material model. In general though, the damage initiation criteria for both material models is strain based and the nature of the damage evolution is assumed to be a function of the effective stress, and the equivalent plastic strain:
\begin{equation}
D=D\left(\bar{\sigma}_{ij},\bar{\epsilon}^{pl}\right)\label{eqn:const8}
\end{equation}



etc..
