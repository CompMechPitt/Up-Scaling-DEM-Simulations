\subsection{Concrete Damaged-Plasticity}
%The CDM model used in this investigation is the concrete damaged plasticity
%model that is implemented in ABAQUS, which is based on the plastic-damage
%model for concrete proposed by \citet{lubliner_plastic-damage_1989} and further
%developed by \citet{lee_plastic-damage_1998} for cyclic loading. The general
%CDM theory considers the stiffness degradation of the material by
%modifying the elastic stiffness tensor with a damage variable. The
%damage variable can be scalar or tensorial in nature, depending on
%the anisotropy of damage evolution. In this investigation, the isotropic
%case is considered and so a scalar damage variable becomes sufficient.

%The parameters of the CDM that were calibrated from DEM simulations
%characterize the elastic and the plastic behavior and the damage evolution.
%Young's modulus and Poisson's ratio parameterize the elastic behavior.
For the concrete damaged-plasticity model, the plastic behavior of the material is largely determined by the
parameters of the hardening rule. Different functions were used
to characterize hardening in tension and compression. The shape of
the curves were based on the laboratory data in \citet{wahalathantri_material_2011} and the aim was to mimic the curves with the least number of
parameters. The compressive hardening rule, $\sigma_{c}\left(\bar{\epsilon}^{in}\right)$,
was approximated using a quadratic function, as shown in Figure \ref{fig:conccomp}. The
quadratic function requires three parameters. It was found to be useful
to manipulate the standard quadratic equation form to allow for the
three parameters to have a physical meaning. This form of the approximation
becomes quite useful for the parameter estimation when applying bounding
limits:

\begin{equation}
\sigma_{c}\left(\bar{\epsilon}^{in}\right)=\frac{\sigma_{c}^{iy}-\sigma_{c}^{p}}{\left(\epsilon_{c}^{pp}\right)^{2}}\left(\bar{\epsilon}^{in}-\epsilon_{c}^{pp}\right)^{2}+\sigma_{c}^{p}\label{eqn:param2-1}
\end{equation}

