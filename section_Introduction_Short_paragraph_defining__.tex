\section{Introduction}
Short paragraph defining homogenization and how it can be used to development a better understanding of rock mechanics.  Statement about gap in the literature addressed in this article, i.e., not yet applied to HF.

Discussion of the natural fractured rocks as a multiscale material.

Brief discussion of literature describing multiscale (upscaling)  modelling methods.  Concurrent vs hierarchical multiscale analysis.  See my review article on my website and \cite{Gracie_2011}. Justify that a hierarchical approach is preferable here, do to long simulated time; concurrent approach is to restrictive in terms of time step size. Also justify/explain selection of homogenization instead of an alternative method of upscaling.

Discussion of literature around modeling using homogenization.
\begin{itemize}
\item Continuum to continuum and DEM to Continuum. rigid versus deformable particals
\item Analogy to debonding of particle -> note lack of contact.
\item What are the common characteristics of the models.
\item What are the unique characteristics of the models.  
\end{itemize} 

Overview of DEM literature with a focus on HF

test

Outline of the article