\section{Homogenization Approach}

The main objective of homogenizing DEM simulations is to be able to
describe the macroscopic behavior of the discontinuous medium in terms
of a standard more computationally efficient continuum model. The
homogenization algorithms used herein are based on the methods developed
by \citet{daddetta_particle_2004} and \citet{wellmann_homogenization_2008}. In this
homogenization process, the resultant inter-block contact forces and
block displacement from the DEM simulations are converted to average
stresses and strains for a continuum.

For the homogenization procedure to yield meaningful results, it should
be applied to a Representative Elementary Volume (REV). The exact size of the REV depends on the geometry
and mechanical properties of the DEM model. For the homogenization
approach to hold, the REV of size $d$ within a system with a characteristic
length $D$ and consisting of blocks with a characteristic diameter
$\delta$, must subscribe to the following scale separation \cite{wellmann_homogenization_2008}. 
\begin{equation}
D\gg d\gg\delta\label{eqn:hom1a}
\end{equation}


To avoid a lengthy discussion of the statistics required to formally define an REV for a particular DFN, an REV is assumed for the purposes of this investigation. The REV is chosen to be a circular subsection of the DEM model. This representative subsection aims to be as large as possible, while being a sufficiently small REV so as to reduce the influence of boundary effects.

Given a circular REV, due to the discontinuous nature of the DEM
simulations, the circular REV cannot be used directly. Because the
calculated displacements and contact forces from the DEM are known
at the block edges, the homogenization domain boundary must follow
the block boundaries. In order to define a homogenization domain based
on the REV, but subscribing to the block boundaries, the homogenization
domain is taken to be the domain defined by the outside boundaries
of the blocks that intersect the REV boundary, as illustrated in figure \ref{fig:homoarea}.
The homogenization domain is characterized by a series of block corners
which are identified for the initial (zero strain) state. These corners
continue to define the homogenization domain once deformation occurs,
allowing for a consistent homogenization domain definition as the
model is deformed. The algorithm that was developed to assess the
homogenization domain boundary is presented as follows: 

\begin{enumerate}
\item \label{hli:1} Identify all blocks that lie on the REV boundary. 
\item \label{hli:2} Identify all blocks that lie completely outside the
REV boundary. 
\item \label{hli:3} Find all contacts corresponding to the intersection
between blocks in \ref{hli:1} and \ref{hli:2}. 
\item \label{hli:4} Find all corners corresponding to the contacts in \ref{hli:3}. 
\item \label{hli:5} Find all contacts of blocks in \ref{hli:1}. 
\item \label{hli:6} Find all blocks corresponding to the intersection of
contacts in \ref{hli:3} and \ref{hli:5}. 
\item \label{hli:7} Find all corners corresponding to blocks in \ref{hli:6}. 
\item \label{hli:8} Find intersection of corners in \ref{hli:4} and \ref{hli:7}. 
\item \label{hli:9} Find corners that initially (at zero strain) coincided
with the corners in \ref{hli:8} 
\item \label{hli:10} Find union of corners in \ref{hli:8} and \ref{hli:9}. 
\item \label{hli:11} Order corners from \ref{hli:10} such that they form
the boundary of the homogenization area
\end{enumerate}

\begin{enumerate}
\item This is the first item. \label{1} 
\item This is the second item referencing the item \ref{1}. \label{2} 
\end{enumerate}
