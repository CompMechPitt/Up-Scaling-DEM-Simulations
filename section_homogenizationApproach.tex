\section{Homogenization Approach}

The main objective of homogenizing DEM simulations is to be able to describe the macroscopic behavior of the discontinuous medium in terms
of a standard more computationally efficient continuum model. The homogenization algorithms used herein are based on the methods developed
by \citet{daddetta_particle_2004} and \citet{wellmann_homogenization_2008}. In this homogenization process, the resultant inter-block contact forces and block displacement from the DEM simulations are converted to average stresses and strains for a continuum.

For the homogenization procedure to yield meaningful results, it should be applied to a Representative Elementary Volume (REV). The exact size of the REV depends on the geometry and mechanical properties of the DEM model. For the homogenization approach to hold, the REV of size $d$ within a system with a characteristic length $D$ and consisting of blocks with a characteristic diameter $\delta$, must subscribe to the following scale separation \citep{wellmann_homogenization_2008}. 

\begin{equation}
D\gg d\gg\delta\label{eqn:hom1a}
\end{equation}

Given a circular REV, due to the discontinuous nature of the DEM simulations, the circular REV cannot be used directly. Because the
calculated displacements and contact forces from the DEM are known at the block edges, the homogenization domain boundary must follow
the block boundaries. In order to define a homogenization domain based on the REV, but subscribing to the block boundaries, the homogenization
domain is taken to be the domain defined by the outside boundaries of the blocks that intersect the REV boundary, as illustrated in Figure \ref{fig:homoarea}. The homogenization domain is characterized by a series of block corners which are identified for the initial (zero strain) state. These corners continue to define the homogenization domain once deformation occurs, allowing for a consistent homogenization domain definition as the model is deformed. 

