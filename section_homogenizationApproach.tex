\section{Homogenization Approach}

The main objective of up-scaling DEM simulations is to be able to describe the  behavior of the discontinuous medium in terms of a more computationally efficient continuum model. The homogenization algorithms used herein to determine the average stress-strain behaviour, $\left<\boldsymbol{\sigma}}\right>$-$\left<\boldsymbol{\epsilon}}\right>$, of the RVE from the microscale displacements $\mathbf{u}^m$, strain $\boldsymbol{\epsilon}^m$, and stresses $\boldsymbol{\sigma}^m$ based on the methods developed by \citet{daddetta_particle_2004} and \citet{wellmann_homogenization_2008}. In this homogenization process, the resultant inter-block contact forces and block displacement from the DEM simulations are converted to average stresses and strains.

For the homogenization procedure to yield meaningful results, it should be applied to a Representative Elementary Volume (REV). The exact size of the REV depends on the geometry and mechanical properties of the DEM model. For the homogenization approach to hold, the REV of size $d$ within a system with a characteristic length $D$ and consisting of blocks with a characteristic diameter $\delta$, must satisfy scale separation: $D\gg d\gg\delta$ \citep{wellmann_homogenization_2008}. 

In the following sections, all deformations are assumed to be small, such that there is no need to differentiate between the deformed and undeformed configurations.  

We begin by defining a $L\times L$ square DEM simulation domain over which mixed-boundary conditions will be applied. The RVE is taken as a circular domain of radius $R$, $2R<L$.  The RVE is taken to be a subdomain of the actual DEM simulation domain to eliminate any boundary effects.  As will be seen below, it is convenient to take the boundary of the domain used for homogenization as a slightly large domain encompassing the RVE boundary. The boundary of the homogenization domain, denoted as $\Gamma_h$, is defined by the outer edges of the deformable blocks, i.e., the cohesive/contact surfaces between deformable blocks, which intersect a circle of radius $R$ located in the center of the DEM simulation domain. Let homogenization domain, the domain bounded by $\Gamma_h$, be denoted by $\Omega_h$. These definitions of illustrated in Figure \ref{fig:homoarea} for a $10$m $\times$ $10$m DEM domain, where the deformable blocks are defined through a Voronoi tessellation. The radius of the RVE domain is $2.5$m.  It can be seen that the actual domain used for homogenization is non-circular and larger then the RVE domain. 


