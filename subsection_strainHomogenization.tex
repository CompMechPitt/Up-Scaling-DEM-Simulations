\subsection{Strain Homogenization}

The derivation for the homogenized strain tensor, $\langle\boldsymbol{\epsilon}\rangle$
begins in a similar manner to the homogenized stress tensor derivation
with the familiar definition of the spatial average:

\begin{equation}
\langle\boldsymbol{\epsilon}\rangle=\frac{1}{A^{h}}\int_{\Omega^{h}}\boldsymbol{\epsilon}dA\label{eqn:strain2}
\end{equation}
Again, the homogenization domain integral needs to be decomposed into
domain integrals of the block subdomain and the void subdomain due
to the inherently different behviour of the block strain and the domain
strain:

\begin{equation}
\langle\boldsymbol{\epsilon}\rangle=\frac{1}{A^{h}}\left[\int_{\Omega_{b}}\boldsymbol{\epsilon}dA+\int_{\Omega_{v}}\boldsymbol{\epsilon}dA\right]\label{eqn:strain3}
\end{equation}
At this point, it becomes convenient to assume a small displacement
formulation of strain. This displacement assumtion limits the appliciability
of the strain homogenization, but in the context of large scale geomechanics,
this assumtion becomes reaonable. As such, the linear infinitesimal
strain tensor can be written in terms of dispalcements, $\mathbf{u}$:

\begin{equation}
\boldsymbol{\epsilon}=\frac{1}{2}\left[\nabla\mathbf{u}+\left(\nabla\mathbf{u}\right)^{T}\right]\label{eqn:strain1}
\end{equation}
The following relationship, which is derived from the divergence theorem,
allows for the simplification of the domain integral to a boundary
integral based on the displacements and the boundary outward normals,
$\mathbf{n}$ (Wellman et al., 2008):

\begin{equation}
\int_{\Omega}\left[\nabla\mathbf{u}+\left(\nabla\mathbf{u}\right)^{T}\right]dA=\oint_{\Gamma}\left[\mathbf{u}\otimes\mathbf{n}+\mathbf{n}\otimes\mathbf{u}\right]d\Gamma\label{eqn:strain1-1}
\end{equation}
Substitution of \ref{eqn:strain1} and \ref{eqn:strain1-1} into \ref{eqn:strain3}
allows one to write the homogenized strain tensor as follows:

\begin{equation}
\langle\boldsymbol{\epsilon}\rangle=\frac{1}{2A^{h}}\left[\oint_{\Gamma^{b}}\left[\mathbf{u}^{b}\otimes\mathbf{n}^{b}+\mathbf{n}^{b}\otimes\mathbf{u}^{b}\right]d\Gamma+\oint_{\Gamma^{v}}\left[\mathbf{u}^{v}\otimes\mathbf{n}^{v}+\mathbf{n}^{v}\otimes\mathbf{u}^{v}\right]d\Gamma\right]\label{eqn:strain5}
\end{equation}


Here, one can again take advantage of the discontinuous nature of
the DEM simulations, allowing for the continuous boundary integrals
to be written as a summation over both the block boundary with $N_{b}$boundary
segments and the void boundary with $N_{v}$boundary segments:

\begin{equation}
\langle\boldsymbol{\epsilon}\rangle=\frac{1}{2A^{h}}\left[\sum_{i=1}^{N_{b}}\left[\mathbf{u}_{i}^{b}\otimes\mathbf{n}_{i}^{b}+\mathbf{n}_{i}^{b}\otimes\mathbf{u}_{i}^{b}\right]L_{i}^{b}+\sum_{i=1}^{N_{v}}\left[\mathbf{u}_{i}^{v}\otimes\mathbf{n}_{i}^{v}+\mathbf{n}_{i}^{v}\otimes\mathbf{u}_{i}^{v}\right]L_{i}^{v}\right]\label{eqn:strain6}
\end{equation}


where $\mathbf{u}_{i}^{b}$and $\mathbf{u}_{i}^{v}$represent the
average displacement along the $i^{th}$boundary segment of the block
boundary and the void boundary respectively. Furthermore,

\begin{equation}
\langle\boldsymbol{\epsilon}\rangle=\frac{1}{2A^{h}}\sum_{i=1}^{N}\left[\mathbf{u}^{i}\otimes\mathbf{n}^{i}+\mathbf{n}^{i}\otimes\mathbf{u}^{i}\right]L^{i}\label{eqn:strain7}
\end{equation}
