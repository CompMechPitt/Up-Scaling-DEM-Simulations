\subsection{Strain Homogenization}
The derivation for the homogenized strain tensor starts much like the homogenized stress tensor derivation with the familiar definition of the spatial average:

\begin{equation}
\label{eqn:strain2}
\langle \boldsymbol{\epsilon} \rangle = 
\frac{1}{A} \int_\Omega \boldsymbol{\epsilon} {dA}
\end{equation}

\begin{equation}
\label{eqn:strain3}
\langle \boldsymbol{\epsilon} \rangle = 
\frac{1}{A} \bigg \lbrack {\int_{\Omega_{b}} \boldsymbol{\epsilon} { dA} + 
\int_{\Omega_{v}} \boldsymbol{\epsilon} {dA}} \bigg \rbrack
\end{equation}

\begin{equation}
\label{eqn:strain4}
\langle \boldsymbol{\epsilon} \rangle = 
\frac{1}{2 A} \bigg \lbrack {\int_{\Omega_{b}} \left[ \nabla \mathbf{u} + 
\left( \nabla \mathbf{u} \right)^{T} \right] {dA} +
\int_{\Omega_{v}} \left[ \nabla \mathbf{u} + 
\left( \nabla \mathbf{u} \right)^{T} \right] {dA}} \bigg \rbrack
\end{equation}

\begin{equation}
\label{eqn:strain5}
\langle \boldsymbol{\epsilon} \rangle = 
\frac{1}{2 A} \bigg \lbrack {\oint_{\Gamma_{b}} \left[ \mathbf{u} \otimes \mathbf{n} + 
\mathbf{n} \otimes \mathbf{u} \right] {d \Gamma} +
\oint_{\Gamma_{v}} \left[ \mathbf{u} \otimes \mathbf{n} + 
\mathbf{n} \otimes \mathbf{u} \right] { d \Gamma} \bigg \rbrack
\end{equation}

\begin{equation}
\label{eqn:strain6}
\langle \boldsymbol{\epsilon} \rangle = 
\frac{1}{2 A} \bigg \lbrack {\sum_{i=1}^{N_{rb}} \left[ \mathbf{u}_{rb}^i \otimes \mathbf{n}_{rb}^i + 
\mathbf{n}_{rb}^i \otimes \mathbf{u}_{rb}^i \right] {L_{rb}^i} +
\sum_{i=1}^{N_{db}} \left[ \mathbf{u}_{db}^i \otimes \mathbf{n}_{db}^i + 
\mathbf{n}_{db}^i \otimes \mathbf{u}_{db}^i \right] {L_{db}^i}} \bigg \rbrack
\end{equation}






\begin{equation}
\label{eqn:strain1}
\boldsymbol{\epsilon} = 
\frac{1}{2}  \left[ \nabla \mathbf{u} + \left( \nabla \mathbf{u} \right)^{T} \right]
\end{equation}