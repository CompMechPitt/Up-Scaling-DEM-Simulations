\subsection{Strain Homogenization}

The derivation for the homogenized strain tensor, $\left< \boldsymbol{\epsilon}\right>$, begins in a similar manner to the homogenized stress tensor derivation with the familiar definition of the spatial average:

\begin{equation}
\langle\boldsymbol{\epsilon}\rangle=\frac{1}{A^{h}}\int_{\Omega^{h}}\boldsymbol{\epsilon}^m dA\label{eqn:strain2}
\end{equation}

At this point, it becomes convenient to assume a small displacement formulation of strain. This displacement assumption limits the applicability of the strain homogenization, but in the context of large scale geomechanics, this assumption remains reasonable. As such, the linear infinitesimal strain tensor can be written in terms of the displacement vector, $\mathbf{u}^m$:

\begin{equation}
\boldsymbol{\epsilon}^m=\frac{1}{2}\left[\nabla\mathbf{u}^m+\left(\nabla^\top \mathbf{u}^m\right)\right]\label{eqn:strain1}
\end{equation}

The above integral can be converted to a following boundary integral using the divergence theorem:

\begin{equation}
\langle\boldsymbol{\epsilon}\rangle=\frac{1}{2A^{h}}\oint_{\Gamma^{h}}\left[\mathbf{u}^m\otimes\mathbf{n}+\mathbf{n}\otimes\mathbf{u}^m\right]d\Gamma\label{eqn:strain5-1}
\end{equation}

where $\mathbf{n}$ is the outward pointing normal to $\Gamma_h$

When $\Gamma_h$ is defined by a set of line segments over which the displacement is also linear, the boundary integral can be rewritten as a summation over each of the $N$ boundary segments. Let $\bar{\mathbf{u}}^m_{I}$ denote the average displacement along the $I^{th}$ boundary segment of the homogenization boundary, which is calculated as the average of the two nodal displacements defining the boundary of each segment. Let $\mathbf{n}_{I}$ represent the outward pointing normal to the $I^{th}$ boundary segment on the homogenization boundary.  Let the length of boundary segment $I$ be denoted by $L_{I}$. The homogenized strain can be rewritten as

\begin{equation}
\langle\boldsymbol{\epsilon}\rangle=\frac{1}{2A^{h}}\sum_{I=1}^{N}\left[\bar{\mathbf{u}}^m_{I}\otimes\mathbf{n}_{I}+\mathbf{n}_{I}\otimes\bar{\mathbf{u}}^m_{I}\right]L_{I}\label{eqn:strain7}
\end{equation}
