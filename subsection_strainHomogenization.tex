\subsection{Strain Homogenization}

The derivation for the homogenized strain tensor, $\langle\boldsymbol{\epsilon}\rangle$
begins in a similar manner to the homogenized stress tensor derivation
with the familiar definition of the spatial average:

\begin{equation}
\langle\boldsymbol{\epsilon}\rangle=\frac{1}{A^{h}}\int_{\Omega^{h}}\boldsymbol{\epsilon}dA\label{eqn:strain2}
\end{equation}
Again, the homogenization domain integral needs to be decomposed into
domain integrals of the block subdomain and the void subdomain due
to the inherently different behviour of the block strain and the domain
strain:

\begin{equation}
\langle\boldsymbol{\epsilon}\rangle=\frac{1}{A^{h}}\left[\int_{\Omega_{b}}\boldsymbol{\epsilon}dA+\int_{\Omega_{v}}\boldsymbol{\epsilon}dA\right]\label{eqn:strain3}
\end{equation}
At this point, it becomes convenient to assume a small displacement
formulation of strain. This displacement assumtion limits the appliciability
of the strain homogenization, but in the context of large scale geomechanics,
this assumtion becomes reaonable. As such, the linear infinitesimal
strain tensor can be written in terms of the displacement vector, $\mathbf{u}$:

\begin{equation}
\boldsymbol{\epsilon}=\frac{1}{2}\left[\nabla\mathbf{u}+\left(\nabla\mathbf{u}\right)^{T}\right]\label{eqn:strain1}
\end{equation}
The following relationship, which is derived from the divergence theorem,
allows for the simplification of the domain integral to a boundary
integral based on the displacements and the boundary outward normals,
$\mathbf{n}$ (Wellman et al., 2008):

\begin{equation}
\int_{\Omega}\left[\nabla\mathbf{u}+\left(\nabla\mathbf{u}\right)^{T}\right]dA=\oint_{\Gamma}\left[\mathbf{u}\otimes\mathbf{n}+\mathbf{n}\otimes\mathbf{u}\right]d\Gamma\label{eqn:strain1-1}
\end{equation}
Substitution of \ref{eqn:strain1} and \ref{eqn:strain1-1} into \ref{eqn:strain3}
allows one to write the homogenized strain tensor as follows, where
$\mathbf{u}^{b}$and $\mathbf{u}^{v}$ represent the average displacements
along the block and void boundaries respectively, and $\mathbf{n}^{b}$and
$\mathbf{n}^{v}$represent the outward normals of the block and void
boundaries repsectively:

\begin{equation}
\langle\boldsymbol{\epsilon}\rangle=\frac{1}{2A^{h}}\left[\oint_{\Gamma^{b}}\left[\mathbf{u}^{b}\otimes\mathbf{n}^{b}+\mathbf{n}^{b}\otimes\mathbf{u}^{b}\right]d\Gamma+\oint_{\Gamma^{v}}\left[\mathbf{u}^{v}\otimes\mathbf{n}^{v}+\mathbf{n}^{v}\otimes\mathbf{u}^{v}\right]d\Gamma\right]\label{eqn:strain5}
\end{equation}


If one considers the strain behaviour of the block boundary and the
void boundary, the displacement and normal of a particular segment
will be identical regardless of weather it is a block boundary segment
or a void boundary segment. As such, the two boundary integrals can
be combined to yeild the following:

\begin{equation}
\langle\boldsymbol{\epsilon}\rangle=\frac{1}{2A^{h}}\oint_{\Gamma^{h}}\left[\mathbf{u}\otimes\mathbf{n}+\mathbf{n}\otimes\mathbf{u}\right]d\Gamma\label{eqn:strain5-1}
\end{equation}


Here, one can again take advantage of the discontinuous nature of
the DEM simulations, allowing for the continuous boundary integrals
to be written as a summation over the boundary with $N$ boundary segments,
where $\mathbf{u}_{i}$ represents the average displacement along
the $i^{th}$ boundary segment on the homogenization boundary. This conversion assumes that the boundary segment is linear and the displacement field along the boundary segment is also linear. The
average displacement along a boundary segment is thus calculated as a linear
average of the two nodal displacements defining the boundary segment.
Furthermore, $\mathbf{n}_{i}$ represents the outward normal of
the $i^{th}$ boundary segment on the homogenization boundary with
a length of $L_{i}$:

\begin{equation}
\langle\boldsymbol{\epsilon}\rangle=\frac{1}{2A^{h}}\sum_{i=1}^{N}\left[\mathbf{u}_{i}\otimes\mathbf{n}_{i}+\mathbf{n}_{i}\otimes\mathbf{u}_{i}\right]L_{i}\label{eqn:strain7}
\end{equation}
