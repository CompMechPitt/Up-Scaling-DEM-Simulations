\subsection{Strain Homogenization}

The derivation for the homogenized strain tensor, $\langle\boldsymbol{\epsilon}\rangle$, begins in a similar manner to the homogenized stress tensor derivation with the familiar definition of the spatial average:

\begin{equation}
\langle\boldsymbol{\epsilon}\rangle=\frac{1}{A^{h}}\int_{\Omega^{h}}\boldsymbol{\epsilon}dA\label{eqn:strain2}
\end{equation}

At this point, it becomes convenient to assume a small displacement formulation of strain. This displacement assumption limits the applicability of the strain homogenization, but in the context of large scale geomechanics, this assumption remains reasonable. As such, the linear infinitesimal strain tensor can be written in terms of the displacement vector, $\mathbf{u}$:

\begin{equation}
\boldsymbol{\epsilon}=\frac{1}{2}\left[\nabla\mathbf{u}+\left(\nabla\mathbf{u}\right)^{T}\right]\label{eqn:strain1}
\end{equation}

The following relationship, which is derived from the divergence theorem, allows for the simplification of the domain integral to a boundary
integral based on the displacement vectors and the boundary outward normals, $\mathbf{n}$ (Wellman et al., 2008):

\begin{equation}
\int_{\Omega}\left[\nabla\mathbf{u}+\left(\nabla\mathbf{u}\right)^{T}\right]dA=\oint_{\Gamma}\left[\mathbf{u}\otimes\mathbf{n}+\mathbf{n}\otimes\mathbf{u}\right]d\Gamma\label{eqn:strain1-1}
\end{equation}

If one considers the strain behaviour of the block boundary and the void boundary, the average displacement and normal of a particular segment
will be identical regardless of whether it is a block boundary segment or a void boundary segment. Substitution of \ref{eqn:strain1-1} into \ref{eqn:strain2} allows one to write the homogenized strain tensor as follows:

\begin{equation}
\langle\boldsymbol{\epsilon}\rangle=\frac{1}{2A^{h}}\oint_{\Gamma^{h}}\left[\mathbf{u}\otimes\mathbf{n}+\mathbf{n}\otimes\mathbf{u}\right]d\Gamma\label{eqn:strain5-1}
\end{equation}


Here, one can again take advantage of the discontinuous nature of the DEM simulations, allowing for the continuous boundary integrals to be written as a summation over the boundary with $N$ boundary segments, where $\mathbf{u}_{i}$ represents the average displacement along the $i^{th}$ boundary segment on the homogenization boundary. This conversion assumes that the boundary segment is linear and the displacement field along the boundary segment is also linear. The average displacement along a boundary segment is thus calculated as a linear average of the two nodal displacements defining the boundary segment. Furthermore, $\mathbf{n}_{i}$ represents the outward normal of the $i^{th}$ boundary segment on the homogenization boundary with a length of $L_{i}$:

\begin{equation}
\langle\boldsymbol{\epsilon}\rangle=\frac{1}{2A^{h}}\sum_{i=1}^{N}\left[\mathbf{u}_{i}\otimes\mathbf{n}_{i}+\mathbf{n}_{i}\otimes\mathbf{u}_{i}\right]L_{i}\label{eqn:strain7}
\end{equation}
