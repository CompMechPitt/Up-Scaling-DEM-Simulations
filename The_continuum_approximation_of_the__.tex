The continuum approximation of the stress fields shows a good match to the smoothed DEM stress fields. More importantly, the load at failure for the two models are quite close. The DEM simulation failed at $11.2 MPa$, while the CDM simulation failed at $11.5 MPa$, a $~3\%$ error considered to be not only negligible in the context of geological uncertainty but acceptable in terms of the computational savings. This agreement of the two models both in terms of the stress distribution and the failure load shows a high degree of success for the up-scaling framework. 

An additional comparison of the surface deflection where the load was applied is presented in Figure \ref{fig:surfacedeflection}. Again, the behaviour of the two models is similar, with downward displacement occurring where the load is applied, upwards displacement towards the slope on the left and negligible displacement towards the right model boundary. Some divergence from  the DEM results can be observed in the CDM approximation where sharp changes in the profile gradient occur; this arises partly from CDM model limitations and partly because the scale of the deviation is similar to the REV scale, which is the limiting case. For larger scales, the error will be smaller.