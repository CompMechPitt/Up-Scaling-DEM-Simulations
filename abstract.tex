Pre-existing fractures significantly influence the geomechanical response of the rock mass at the reservoir scale.  Discrete Element Methods (DEM) are often used to explicitly model the mechanics of Naturally Fractured Rock (NFR); however, they are too computationally prohibitive for reservoir-scale problems. A DEM up-scaling framework is presented here to facilitate estimating a representative parameter set for continuum constitutive models that captures the salient feature of NFR behaviour. Up-scaling is achieved by matching homogenized DEM stress-strain curves from multiple load paths to those of continuum constitutive models using a Particle Swarm Optimization (PSO) algorithm followed by a Damped Least-Squares (DLS) algorithm. The effectiveness of the framework is demonstrated by up-scaling a DEM model of a NFR to a Drucker-Prager damage-plasticity model; the up-scaled model is shown to capture well the effect of confinement on the the yielding and sliding of natural fractures in the rock mass. The up-scaling methodology is further verified through a case study on a naturally fractured granite slope in which the top surface is loaded until failure. The up-scaled continuum model is shown to compare quite well to Direct Numerical Simulation (DNS) and requires two orders of magnitude less computational time.