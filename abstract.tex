Pre-existing fractures significantly influence the geomechanical response of the rock mass at reservoir scale. However, attempting to capture the constitutive response of the Naturally Fractured Rockmass (NFR) in a laboratory context is impractical because the Representative Elementary Volume (REV) samples are very large and NFR is difficult to handle when confining stress is removed. To address this limitation of continuum models, Discrete Element Methods (DEM) are used to explicitly model the mechanics of the discontinuities of NFR to capture the constitutive response of the rock mass indirectly. However, due to the large number of degrees of freedom in DEM simulations and the requirement of small times steps, the application of realistic DEM simulations to reservoir-scale problems problems is often computationally prohibitive. 

In order to avoid these computational costs associated with full-scale DEM simulations, an up-scaling framework is presented here to facilitate estimating a representative parameter set for a continuum constitutive model that accurately simulates the NFR behaviour. Up-scaling is achieved by matching homogenized stress-strain curves from REV-scale DEM simulations to single element continuum models using a Particle Swarm Optimization (PSO) algorithm followed by a Damped Least-Squares (DLS) algorithm. A Drucker-Prager plasticity model with ductile damage was implemented as the constitutive model to empirically capture the effect of the degradation of the NFR integrity due to the yielding and sliding of natural fractures in the rock mass.

The up-scaling methodology is demonstrated through a case study on a naturally fractured granite slope in which the top surface is loaded until failure. A Direct Numerical Simulation (DNS) using the estimated continuum parameter set compares very well with the DEM model but requires two orders of magnitude less computational time.