\section{Plasticity Based Damage Mechanics}

Damage mechanics allow for the 

Plasticity based damage mechanics models attempt to incorporate the
theories of plasticy and damage machanics into a unified approach
to the damage evolution and constituive relationships \cite{zhang_continuum_2010}. These 




The constitutive stress-strain relationship including a scalar damage
parameter, $\mathbf{D}$ can be written as follows: 
\begin{equation}
\boldsymbol{\sigma}=(1-\mathbf{D})\mathbf{E}:\boldsymbol{\epsilon^{el}}\label{eqn:const3}
\end{equation}


For simplicity, the damaged elastic stiffness is described as the
reduced stiffness due to the damage: 
\begin{equation}
\mathbf{E^{d}}=(1-\mathbf{D})\mathbf{E}\label{eqn:const4}
\end{equation}


Substituting \ref{eqn:const1} and \ref{eqn:const4} into \ref{eqn:const3}
results in the following: 
\begin{equation}
\boldsymbol{\sigma}=\mathbf{E^{d}}:(\boldsymbol{\epsilon}-\boldsymbol{\epsilon}^{pl})\label{eqn:const5}
\end{equation}


Using the 'usual notions of CDM' (find
reference), the effective stress. $\boldsymbol{\bar{\sigma}}$, can
be defined as: 
\begin{equation}
\boldsymbol{\bar{\sigma}}=\mathbf{E}:(\boldsymbol{\epsilon}-\boldsymbol{\epsilon}^{pl})\label{eqn:const6}
\end{equation}


Such that the cauchy stress tensor can be realted to the effective
stress tensor as follows: 
\begin{equation}
\boldsymbol{\sigma}=(1-\mathbf{D})\boldsymbol{\bar{\sigma}}\label{eqn:const7}
\end{equation}


The nature of the damage evolution is assumed to be a function of
the effective stress and the equivalent plastic strain, $\boldsymbol{\bar{\epsilon}^{pl}}$:
\begin{equation}
\mathbf{D}=\mathbf{D}(\boldsymbol{\bar{\sigma}},\boldsymbol{\bar{\epsilon}^{pl}})\label{eqn:const8}
\end{equation}

\subsubsection{misc}
Damage plasticity model based on Strain can be decomposed into elastic
and plastic components: 
\begin{equation}
\boldsymbol{\epsilon}=\boldsymbol{\epsilon}^{el}+\boldsymbol{\epsilon}^{pl}\label{eqn:const1}
\end{equation}


Taking the time derivative gives the decomposition of the strain rate,
$\boldsymbol{\dot{\epsilon}}$ : 
\begin{equation}
\boldsymbol{\dot{\epsilon}}=\boldsymbol{\dot{\epsilon}}^{el}+\boldsymbol{\dot{\epsilon}}^{pl}\label{eqn:const2}
\end{equation}
