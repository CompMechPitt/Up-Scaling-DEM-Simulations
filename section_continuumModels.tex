\section{Plasticity Based Damage Mechanics}

Continuum Damage Mechanics (CDM) is a branch of continuum mechanics that is concerned with modeling the progressive failure and stiffness degradation in solid materials. CDM is traditionally applied to describe the micro-mechanical degradation of metals due to the nucleation and growth of micro-cracks and micro-voids. This micro-mechanical degradations is represented in a CDM model by using macroscopic state variables to represent a spatial average of the effects of this degradation. These state variables used in this context with respect to CDM are known as damage variables. 

A CDM model is comprised of two main components to describe the material degradation: a damage initiation criteria, which dictates the conditions required for the material to first yield, as well as a damage evolution function, which indicates how the damage variables will progress subsequent to the damage initiation. 

The damage variables in a CDM model can take a number of forms. Often, for mathematical and physical simplicity, a single scalar damage variable is used to characterize the the state of damage in the material. In this case, the damage variable takes a value between 0 and 1 to represent the degree of damage to the material, where 0 represents a completely undamaged material (original stiffness) and 1 represents a completely damaged material (0 stiffness). A scalar description of damage does limit the applicability of the CDM model to an isotropically damaged state, which may not be appropriate in some circumstances. More sophisticated CDM models use 2\textsuperscript{nd} and 4\textsuperscript{th} order tensorial representations of the damage variables as well as distinguishing between compressive damage and tensile damage states in order to more accurately characterize anisotropic damage conditions. 

In terms of a constitutive relationship relating the Cauchy stress, $\boldsymbol{\sigma}$, to the elastic strain, $\boldsymbol{\epsilon^{el}}$, one can modify the standard elastic constitutive relationship to include a scalar damage
parameter, $\mathbf{D}$, which can be written as follows: 
\begin{equation}
\boldsymbol{\sigma}=(1-\mathbf{D})\mathbf{E}:\boldsymbol{\epsilon^{el}}\label{eqn:const3}
\end{equation}

Such that the damage variable can be considered as acting a scaling factor for the elastic stiffness, $\mathbf{E}$. From equation \ref{eqn:const3}, it follows that the damaged elastic stiffness, $\mathbf{E^{d}}$, can be described as the degraded stiffness due to the damage: 
\begin{equation}
\mathbf{E^{d}}=(1-\mathbf{D})\mathbf{E}\label{eqn:const4}
\end{equation}

Plasticity based damage mechanics models attempt to incorporate CDM into elasto-plastic mechanical models. 
theories of plasticy and damage machanics into a unified approach to the damage evolution and constituive relationships \cite{zhang_continuum_2010}. These 





Substituting \ref{eqn:const1} and \ref{eqn:const4} into \ref{eqn:const3}
results in the following: 
\begin{equation}
\boldsymbol{\sigma}=\mathbf{E^{d}}:(\boldsymbol{\epsilon}-\boldsymbol{\epsilon}^{pl})\label{eqn:const5}
\end{equation}


Using the 'usual notions of CDM' (find
reference), the effective stress. $\boldsymbol{\bar{\sigma}}$, can
be defined as: 
\begin{equation}
\boldsymbol{\bar{\sigma}}=\mathbf{E}:(\boldsymbol{\epsilon}-\boldsymbol{\epsilon}^{pl})\label{eqn:const6}
\end{equation}


Such that the cauchy stress tensor can be realted to the effective
stress tensor as follows: 
\begin{equation}
\boldsymbol{\sigma}=(1-\mathbf{D})\boldsymbol{\bar{\sigma}}\label{eqn:const7}
\end{equation}


The nature of the damage evolution is assumed to be a function of
the effective stress and the equivalent plastic strain, $\boldsymbol{\bar{\epsilon}^{pl}}$:
\begin{equation}
\mathbf{D}=\mathbf{D}(\boldsymbol{\bar{\sigma}},\boldsymbol{\bar{\epsilon}^{pl}})\label{eqn:const8}
\end{equation}

\subsubsection{misc}
Damage plasticity model based on Strain can be decomposed into elastic
and plastic components: 
\begin{equation}
\boldsymbol{\epsilon}=\boldsymbol{\epsilon}^{el}+\boldsymbol{\epsilon}^{pl}\label{eqn:const1}
\end{equation}


Taking the time derivative gives the decomposition of the strain rate,
$\boldsymbol{\dot{\epsilon}}$ : 
\begin{equation}
\boldsymbol{\dot{\epsilon}}=\boldsymbol{\dot{\epsilon}}^{el}+\boldsymbol{\dot{\epsilon}}^{pl}\label{eqn:const2}
\end{equation}
