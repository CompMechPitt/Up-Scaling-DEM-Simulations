There are additional minor sources of error from the homogenization algorithms that do not manifest themselves in this fitted relationship.  In addition, the REV, if too small, would also manifest it's error in the DEM data rather than the fitted response. Furthermore, the global fitting algorithms are not completely exhaustive, so it is possible they do not find the actual globally optimal parameter set, potentially causing some of the error. With the given PSO parameters, up to 2400 sets of simulations are conducted for the global parameter estimation, and successive fitting operations tend to yield results within 1\% deviation. This consistency and large search gives confidence that the estimated parameter set is the globally optimal set. 

In addition to the loading response under the specified confining stresses, DEM simulations under confining stresses of $3MPa$, $6MPa$, $8MPa$ and $10MPa$ were compared to the the CDM model using the previously estimated parameter set in order to see how well the constitutive behaviour was captured in Figure \ref{fig:fitted2}. These simulations are intended show the interpolative ($3MPa$) and extrapolative ($6MPa$, $8MPa$ and $10MPa$) capacity of the fitted parameter set. It can be seen that a strong fit occurs (RMSE of $2.83MPa$) for all of the confining stresses with the error being more prominent for larger degrees of strain.