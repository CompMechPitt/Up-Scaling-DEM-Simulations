There are additional minor sources of error from the homogenization algorithms that do not manifest themselves in this fitted relationship.  In addition, the REV, if too small, would also manifest it's error in the DEM data rather than the fitted response. Furthermore, the global fitting algorithms are not completely exhaustive, so it is possible they do not find the actual globally optimal parameter set, potentially causing some of the error. With the given PSO parameters, up to 2400 sets of simulations are conducted for the global parameter estimation, and successive fitting operations tend to yield results within 1\% deviation. This consistency and large search gives confidence that the estimated parameter set is the globally optimal set. 

In addition to the monotonic loading response, the DEM simulations with the complete strain reversal were compared to the the CDM model using the previously estimated parameter set in order to see how well the stiffness degradation effects were captured in Figure \ref{fig:fitted2}. It can be seen that the stiffness degradation is occuring in the DEM simulation before the peak yield stress is reached, so when the strain is reversed  before the damage initiation criterion is met in the CDM model, the unloading stiffness is the same as the original stiffness (as is the case with the $2MPa$ and $4MPa$ cases) which causes the curves to diverge slightly. However, once the damage initiation criterion is met (as is the case with the $1MPa$ and $0.5MPa$ cases) the unloading curved match with a very strong correlation. This pre-damage unloading deviation limits the applicability of this constitutive model for cyclic loading.