\section{Conclusions}
In this paper, a framework for up-scaling DEM simulations is presented. Up-scaling is achieved by matching homogenized stress-strain curves from REV-scale DEM simulations to single element continuum models using PSO and DLS optimization algorithms. A Drucker-Prager plasticity model with ductile damage was implemented as the CDM model to empirically capture the effect of the degradation of the NFR.

It was shown that the Drucker-Prager model with ductile damage was a reasonable CDM model to represent NFR in a continuum context due to the pressure dependent yield criterion and the triaxiality based damage initiation criterion. When fit to the DEM data, the CDM model showed a very good fit pre-damage, but wasn't able to capture some of the subtle oscillations post-yield to due discontinuous yielding in the NFR.

Most importantly, the DNS showed that with this up-scaling framework, very comparable results to the DEM solution can be obtained with the CDM solution but with 2 orders of magnitude less computational time.

This formulation does have its limitations in that it is only formulated for small strains, it is a purely mechanical formulation, and only 2D is considered. However, the framework is exactly the same for the large strain case, the hydro-mechanically coupled case, and the 3D case. Just the software packages need to be swapped in to facilitate these more complex models, but the up-scaling methodology remains the same. It is speculated that in these more complex cases, the up-scaling from DEM to CDM will yeild even greater computational efficiencies. 











\clearpage