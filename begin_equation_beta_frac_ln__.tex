\begin{equation}
\beta=\frac{\ln\left[\frac{2\alpha}{1+\alpha} \right ]}{\epsilon_c^{pp}}
\label{eqn:param2-2}
\end{equation}

\begin{equation}
\alpha =\frac{2\sigma_c^{p}-\sigma_c^{iy}+2\sqrt{-\sigma_c^p\left(\sigma_c^{iy}-\sigma_c^p \right )}}{\sigma_c^{iy}}
\label{eqn:param2-3}
\end{equation}


Similarly, the Johnson-Cook damage initiation criterion from \ref{eqn:druc8} is described in terms of two material parameters, $D_2$, and $D_3$. These parameters describe the shape of the curve, but do not have any physical meaning. To make setting the bounding limits during the parameter estimation simpler, the Johnson-Cook parameters were written in terms of yield plastic strain at triaxialities of -0.5 and -0.75, $\bar{\epsilon}^{pl}_{f_{-0.5}}$,and $\bar{\epsilon}^{pl}_{f_{-0.75}}$ respectively:

\begin{equation}
D_2=\frac{\left(\bar{\epsilon}^{pl}_{y_{-0.5}}\right)^3}{\left(\bar{\epsilon}^{pl}_{y_{-0.75}}\right)^2}
\label{eqn:dparam8}
\end{equation}

\begin{equation}
D_3=4\ln \left (\frac{\bar{\epsilon}^{pl}_{y_{-0.5}}}{\bar{\epsilon}^{pl}_{y_{-0.75}}}\right )
\label{eqn:dparam9}
\end{equation}

In addition, the damage evolution was parameterized using only the plastic displacement at failure parameter. The complete list of all 11 aformentioned parameters can be found summarized in Table \ref{tab:druckerParameters}.

\begin{table}[]
\centering
\caption{Paramater set for Drucker-Prager Material Model with Ductile Damage}
\label{tab:druckerParameters}
\begin{tabular}{@{}cccc@{}}
\end{tabular}
\end{table}

\begin{table}[]
\centering
\caption{Paramater set for Drucker-Prager Material Model with Ductile Damage}
\label{tab:1}
\begin{tabular}{@{}cccc@{}}
\toprule
\multicolumn{2}{c}{\textbf{Parameter Type}}                                                                      & \textbf{Name}                              & \textbf{Symbol}                   \\ \midrule
\multicolumn{2}{c}{\multirow{2}{*}{Elastic}}                                                                     & Young's Modulus                            & $E$                               \\
\multicolumn{2}{c}{}                                                                                             & Poisson's Ratio                            & $\nu$                             \\ \cmidrule{1-2}
\multirow{6}{*}{Plastic} & \multirow{3}{*}{\begin{tabular}[c]{@{}c@{}}Flow Rule /\\ Yield Function\end{tabular}} & Dilation Angle                             & $\psi$                            \\
                         &                                                                                       & Yield Stress Ratio                         & $K$                               \\
                         &                                                                                       & Friction Angle                             & $\beta$                           \\ \cmidrule{2-2}
                         & \multirow{3}{*}{Hardening Rule}                                                       & Initial Compressive Yield Strength         & $\sigma_c^{iy}$                   \\
                         &                                                                                       & Peak Compressive Yield Strength & $\sigma_c^{p}$                    \\
                         &                                                                                       & Strain at Peak Compressive Yield           & $\epsilon_c^{pp}$                 \\ \cmidrule{1-2}
\multirow{3}{*}{Damage}  & \multirow{2}{*}{Initiation}                                                           & Yield Strain at $-0.5$ Triaxiality         & $\bar{\epsilon}^{pl}_{y_{-0.5}}$  \\
                         &                                                                                       & Yield Strain at $-0.75$ Triaxiality        & $\bar{\epsilon}^{pl}_{y_{-0.75}}$ \\ \cmidrule{2-2}
                         & Evolution                                                                             & Plastic Displacement at Failure            & $\bar{u}^{pl}_f$                  \\ \bottomrule
\end{tabular}
\end{table}

\begin{table}[]
\caption{test}
\label{tab:1}
\begin{tabular}{@{}lll@{}}
\end{tabular}
\end{table}