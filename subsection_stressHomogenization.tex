\subsection{Stress Homogenization}

The homogenized Cauchy stress, $\langle\boldsymbol{\sigma}\rangle$, is derived from the definition of the spatial average of the stress, $\boldsymbol{\sigma}$, over the homogenization domain $\Omega^{h}$, with an area $A^{h}$: 
\begin{equation}
\langle\boldsymbol{\sigma}\rangle=\frac{1}{A^{h}}\int_{\Omega^{h}}\boldsymbol{\sigma}dA\label{eqn:stress1}
\end{equation}


As previously mentioned, there exists a necessary distinction between the block subdomain and the void subdomain when dealing with stress. Here, the integration over the homogenization domain can be decomposed into two seperate integrations over the block subdomain and the void subdomain: 

\begin{equation}
\langle\boldsymbol{\sigma}\rangle=\frac{1}{A^{h}}\left[\int_{\Omega^{b}}\boldsymbol{\sigma}dA+\int_{\Omega^{v}}\boldsymbol{\sigma}dA\right]\label{eqn:stress2}
\end{equation}

This distinction is made due to the fact that in a purely mechanical model, there isn't any pressure or stress being retained in the void space, resulting in a negligable contribution to the overall stress state. This allows for the void subdomain integration term to be dropped from the formulation: 

\begin{equation}
\langle\boldsymbol{\sigma}\rangle=\frac{1}{A^{h}}\int_{\Omega^{b}}\boldsymbol{\sigma}dA\label{eqn:stress2a}
\end{equation}

Since the block subdomain is inherently discretized, the integration of the stress over the block subdomain can be written as a summation in the form of a spatially weighted average of the zone stresses, $\boldsymbol{\sigma}_{ij}^{z}$. For $N_{i}^{z}$ zones within each of $N^{b}$ number of blocks in the block subdomain the average stress can be weighted based on the zone area, $A_{ij}^{z}$: 

\begin{equation}
\langle\boldsymbol{\sigma}\rangle=\frac{1}{A^{h}}\sum_{i=1}^{N^{b}}\sum_{j=1}^{N_{i}^{z}}\boldsymbol{\sigma}_{ij}^{z}A_{ij}^{z}\label{eqn:stress4}
\end{equation}

Here, (\ref{eqn:stress4}) represents the final form of the homogenized stress tensor formulation.