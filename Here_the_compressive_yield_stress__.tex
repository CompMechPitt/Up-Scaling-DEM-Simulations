Here, the compressive yield stress, $\sigma_{c}$, is written as function
of the inelastic strain, $\bar{\epsilon}^{in}$, and three additional
parameters. The three parameters are the initial compressive yield
stress ($\sigma_{c}^{iy}$), the peak compressive yield stress ($\sigma_{c}^{p}$),
and the plastic strain at the peak compressive yield stress ($\sigma_{c}^{iy}$).
The physical significance of each of these parameters can be seen
in Fig \ref{fig:conccomp}, where they define the y-intercept and the peak of the curve.

The tensile hardening rule has a fundamentally different behavior
than the compressive hardening rule, and was therefore approximated
using an exponential function (Fig \ref{fig:conctens}). The exponential function required
only two parameters to characterize the curve completely. The first
parameter was the initial tensile yield stress, $\sigma_{t}^{iy}$,
which defines the y-intercept of the curve, while the second parameter
was the tensile yield stress decay parameter,$\lambda$. These parameters
describe the relationship between the tensile yield stress, $\sigma_{t}$,
and the cracking strain, $\bar{\epsilon}^{ck}$, and has the form:

\begin{equation}
\sigma_{t}\left(\bar{\epsilon}^{ck}\right)=\sigma_{t}^{iy}e^{\lambda\bar{\epsilon}^{ck}}\label{eqn:param2}
\end{equation}


