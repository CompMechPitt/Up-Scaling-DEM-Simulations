\subsection{Parameter Estimation Results}

Using a PSO algorithm followed by an LMA optimization, the Drucker-Prager plasticity model with ductile damage was fitted to the post-yield DEM simulation data in order to obtain an optimal parameter set. The stress-strain behaviour of the simulations using the optimal parameter set can be seen alongside the homogenized DEM simulation response in Figure \ref{fig:fitted1}. In this case, it can be see that the CDM fit is strongly correlated to the DEM data. In the DEM data, it can be seen that after the rock has yielded, oscillatory noise is observed in the data. This noise arises from the subsequent fracturing within the rock mass after the ultimate yield stress has been reached. The continuum model cannot account for this behaviour, and the oscilations noted in the CDM data result form dynamic instabilities during the softening. The loading cures are very strongly correlated here before yield showing a very strong capacity of the pressure dependent Drucker-Prager yeild criterion to handle pre-fracture yield at various confining stresses. The deviation in the results occurs more significantly during the post-fracture response, where the fracture points do exhibit slight inconsistencies between the DEM and CDM simulations. 