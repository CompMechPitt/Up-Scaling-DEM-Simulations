This algorithm for defining the homogenization domain was developed
to provide an unambiguous method for assessing a unique homogenization
domain for a given circle radius and location. In steps \ref{hli:1}
and \ref{hli:2}, two mutually exclusive block sets are identified,
which necessarily share contacts. It is the boundary between these
two block sets that is the homogenization domain boundary which is
characterized by this algorithm. Ultimately, the goal of this process
is to identify the corners on this boundary that are on the blocks
that intersect with the REV boundary and not the blocks that are outside
the REV boundary. As such, the corners (step \ref{hli:4}) corresponding
to the contacts (step \ref{hli:3}) between the two sets of blocks
are determined. Step \ref{hli:5} is necessary to help eliminate some
blocks from the set of boundary block which intersect the REV boundary,
but only have contact with blocks inside the homogenization domain
boundary, and thus do not have any overall contribution to the definition.
Step \ref{hli:6} determines this resultant set of boundary blocks
of which every member is connected to the boundary in some capacity.
Finding the set of corners that is mutually shared by these blocks
in step \ref{hli:7} and the boundary contact corners in step \ref{hli:4}
allow for the determination of the initial set of boundary corners
(step \ref{hli:8}).

However, one must also note the potential displacement jumps that
may occur between blocks on the boundary as the model deforms. In
the case where the blocks become physically separated, there exists
a discontinuity along the homogenization boundary as can be seen in
\ref{fig:homoboundary}. These discontinuities along the homogenization
boundary were considered by adding boundary segments to the homogenization
boundary between the corners of the adjacent blocks. As such, steps
\ref{hli:9} and \ref{hli:10} find coincident corners on adjacent
boundary blocks to allow for the blocks to become separated on the
boundary, while still maintaining connectivity along the homogenization
domain boundary. It is also necessary to order the corners (step \ref{hli:11})
as they would appear along the homogenization boundary in order to
define the boundary segments along which integration can be performed.

The homogenization boundary, $\Gamma_{h}$, can be described in terms
of $n$ ordered boundary vertices, $V_{i}^{h}=(x_{i}^{h},y_{i}^{h})$,
representing the $i$th set of vertex coordinates along the boundary,
such that the homogenization area, $A^{h}$, can be calculated using
the following formulation for the area of an arbitrary, non-self-intersecting
polygon(find reference): 
\begin{equation}
A^{h}=\dfrac{1}{2}\sum_{i=1}^{n}x_{i}^{h}(y_{i+1}^{h}-y_{i-1}^{h})\label{eqn:hom1}
\end{equation}


At this point, within the homogenization domain, one must differentiate
between the blocks and the void space between the blocks as they have
fundamentally different behaviour. As such, the homogenization domain
can be decomposed into two mutually exclusive subdomains representing
the voids and blocks, $\Omega^{v}$ and $\Omega^{b}$, respectively.
The total block area associated with the block subdomain, $A^{b}$,
can be assessed as a summation of $m$ block areas within the homogenization
domain, while the individual block area can be assessed in a similar
manner to \ref{eqn:hom1}. For $n^{j}$ block boundary vertices, $V_{i,j}^{b}=(x_{i,j}^{b},y_{i,j}^{b})$
representing the $i$th set of vertex coordinates on the $j$th block,
the total block area can be calculated as: 
\begin{equation}
A^{b}=\dfrac{1}{2}\sum_{j=0}^{m}\sum_{i=1}^{n_{j}}x_{i,j}^{b}(y_{i+1,j}^{b}-y_{i-1,j}^{b})\label{eqn:hom2}
\end{equation}


Assuming that the block domain and the void domain are jointly exhaustive
of the total homogenization domain, the total area associated with
the void subdomain, $A^{v}$, can be written as the difference of
the homogenization domain area and the block subdomain area: 
\begin{equation}
A^{v}=A^{h}-A^{b}\label{eqn:hom3}
\end{equation}
\begin{equation}
A^{v}=\dfrac{1}{2}\sum_{i=1}^{n}x_{i}^{h}(y_{i+1}^{h}-y_{i-1}^{h})-\dfrac{1}{2}\sum_{j=0}^{m}\sum_{i=1}^{n_{j}}x_{i,j}^{b}(y_{i+1,j}^{b}-y_{i-1,j}^{b})\label{eqn:hom4}
\end{equation}
