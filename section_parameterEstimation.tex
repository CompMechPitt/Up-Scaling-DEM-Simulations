\subsection{Parameter Estimation}
%A total of nine parameters were identified as being essential to the definition of the CDM material model. The elastic behavior of the rock mass was described using two parameters, Young’s Modulus, and Poisson’s Ratio, while 5 additional parameters were used to describe the plastic behavior of the rock mass in both tension and compression. The remaining two parameters describe the damage evolution in tension and compression respectively.
Parameter estimation can be described as the process of attempting to obtain an optimal parameter set for a given model, yielding a solution that is closest in fit to the prescribed data. In this paper, the parameter estimation was conducted using calibration algorithms, a subset of optimization which attempts to minimize a least-squares objective function \cite{matott_ostrich:_2008}. Optimization algorithms can often be described in one of two categories, deterministic algorithms (local search) and heuristic algorithms (global search). Deterministic optimization algorithms primarily focus on searching for the optima within the local parameter space by iteratively converging towards a solution. Heuristic optimization algorithms on the other hand explore the entire parameter space approximately and provide an estimate of the global optima. These heuristic techniques are useful in highly non-linear problems, where there are numerous local optima within the prescribed parameter space. In the case where searching the global parameter-space deterministically becomes too computationally demanding, heuristic methods trade optimality, completeness, accuracy and precision for speed (ref). 

A combination of two optimization algorithms were used in this paper to assess the optimal parameter set. An initial heuristic algorithm was applied to search for the approximate global optima, followed by a deterministic algorithm as a local refinement of the optimal parameter set. Particle Swarm Optimization (PSO) was used in this paper as the global heuristic search, while the Levenberg-Marquardt Algorithm (LMA) was used as the local deterministic search. 

The PSO algorithm was developed by \citet{Kennedy} as a byproduct of modeling the cooperative-competitive nature of social behaviour in birds as they flocked searching for food. The PSO algorithm, in a conceptual sense, consists of a series of 'particles' (birds) which 'swarm' through the entire parameter space (sky) searching for the global optima (food) using a combination of individual 'particle' knowledge and global 'swarm' (flock) knowledge.

The parameter estimation algorithm that was used for this investigation is known as the Levenburg-Marquardt Algorithm (LMA) which was proposed by \citet{marquardt_algorithm_1963} and aims to minimize non-linear least-squares problems. This investigation uses an implementation by \citet{matott_ostrich:_2008} in the form of model-independent optimization software, OSTRICH.

The LMA works by iteratively running the CDM model on a single element model, subject to the same boundary conditions as the DEM model, with successive parameter sets that intelligently adapt in order to converge to the DEM data. The axial stress-strain data was taken to be the most indicative of the material response and thus used for the parameter estimation datasets.

%In order to accurately assess the optimal parameter set, the parameter estimation of the CDM material model was conducted in two steps. The first step consisted of a uniaxial tension test to estimate the elastic and tensile plastic properties of the material. The test direction reverses to compression before complete yield of the material in order to assess the tensile damage parameter. 

