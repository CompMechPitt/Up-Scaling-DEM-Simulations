\subsection{Parameter Estimation}
Parameter estimation in the context of this paper is the process of obtaining a parameter set for a CDM model that minimizes the difference between the CDM and the DEM constitutive response. In this paper, the parameter estimation was conducted using calibration algorithms, a subset of optimization which attempts to minimize a least-squares objective function \cite{matott_ostrich:_2008}. Optimization algorithms can often be described in one of two categories, deterministic algorithms (local search) and heuristic algorithms (global search). Deterministic optimization algorithms primarily focus on searching for the optima within the local parameter space by iteratively converging towards a solution. Heuristic optimization algorithms on the other hand explore the entire parameter space approximately and provide an estimate of the global optima. These heuristic techniques are useful in highly non-linear problems, where there are numerous local optima within the prescribed parameter space. In the case where searching the global parameter-space deterministically becomes too computationally demanding, heuristic methods trade completeness and accuracy for speed (ref). 

A combination of two optimization algorithms were used in this paper to assess the optimal parameter set. An initial heuristic algorithm was applied to search for the approximate global optima, followed by a deterministic algorithm as a local refinement of the optimal parameter set. Particle Swarm Optimization (PSO) was used in this paper as the global heuristic search, while the Levenberg-Marquardt Algorithm (LMA) was used as the local deterministic search. 

The PSO algorithm was developed by \citet{Kennedy} as a byproduct of modeling the cooperative-competitive nature of social behaviour in birds as they flocked searching for food. The PSO algorithm, in a conceptual sense, consists of a series of 'particles' (birds) which 'swarm' through the entire parameter space (sky) searching for the global optima (food) using a combination of individual 'particle' knowledge and global 'swarm' (flock) knowledge.

The Levenburg-Marquardt Algorithm (LMA) was proposed by \citet{marquardt_algorithm_1963} which builds off of the work of \citet{levenberg_method_1944}. This calibration algorithm combines a quasi-Newton approach with a conjugate gradient technique in order to efficiently minimize non-linear least-squares problems. 

The parameter estimation works by iteratively running a single element CDM model, subject to the same boundary conditions as the DEM model, with successive parameter sets that intelligently adapt in order to converge to the DEM data. 

This investigation utilizes OSTRICH for the parameter estimation to avoid the need to re-implement the aforementioned algorithms. OSTRICH is a model-independent optimization package written by \citet{matott_ostrich:_2016} which contains implementations of both the PSO and LMA algorithms.

