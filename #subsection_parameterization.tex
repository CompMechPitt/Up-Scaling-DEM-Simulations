\subsection{Model Parameterization}


\subsubsection{Concrete Damaged-Plasticity}

The CDM model used in this investigation is the concrete damaged plasticity
model that is implemented in ABAQUS, which is based on the plastic-damage
model for concrete proposed by Lubliner et al. (1989) and further
developed by Lee and Fenves (1998) for cyclic loading. The general
CDM theory considers the stiffness degradation of the material by
modifying the elastic stiffness tensor with a damage variable. The
damage variable can be scalar or tensorial in nature, depending on
the anisotropy of damage evolution. In this investigation, the isotropic
case is considered and so a scalar damage variable becomes sufficient.
The parameters of the CDM that were calibrated from DEM simulations
characterize the elastic and the plastic behavior and the damage evolution.
Young's modulus and Poisson'��s ratio parameterize the elastic behavior.
The plastic behavior of the material is largely determined by the
parameters of the hardening rule, and different functions were used
to characterize hardening in tension and compression. The shape of
the curves were based on the laboratory data in Wahalathantri et al.
(2011) and the aim was to mimic the curves with the least number of
parameters. The compressive hardening rule,$\sigma_{c}\left(\bar{\epsilon}^{in}\right)$,
was approximated using a quadratic function, as shown in Fig 3. The
quadratic function requires three parameters. It was found to be useful
to manipulate the standard quadratic equation form to allow for the
three parameters to have a physical meaning. This form of the approximation
becomes quite useful for the parameter estimation when applying bounding
limits:

\begin{equation}
\sigma_{c}\left(\bar{\epsilon}^{in}\right)=\frac{\sigma_{c}^{iy}-\sigma_{c}^{p}}{\left(\epsilon_{c}^{pp}\right)^{2}}\left(\bar{\epsilon}^{in}-\epsilon_{c}^{pp}\right)^{2}+\sigma_{c}^{p}\label{eqn:param2-1}
\end{equation}


Fig 3 Compressive hardening rule for the CDM model using three parameters
for a quadratic approximation. 

Here, the compressive yield stress, $\sigma_{c}$, is written as function
of the inelastic strain, �\textmu{} ̃in, and three additional parameters.
The three parameters are the initial compressive yield stress ($\sigma_{c}^{iy}$),
the peak compressive yield stress ($\sigma_{c}^{p}$), and the plastic
strain at the peak compressive yield stress ($\sigma_{c}^{iy}$).
The physical significance of each of these parameters can be seen
in Fig 3, where they define the y-intercept and the peak of the curve.
The tensile hardening rule has a fundamentally different behavior
than the compressive hardening rule, and was therefore approximated
using an exponential function (Fig 4). The exponential function required
only two parameters to characterize the curve completely. The first
parameter was the initial tensile yield stress, $\sigma_{t}^{iy}$,
which defines the y-intercept of the curve, while the second parameter
was the tensile yield stress decay parameter,$\lambda$. These parameters
describe the relationship between the tensile yield stress, $\sigma_{t}$,
and the cracking strain, $\bar{\epsilon}^{ck}$, and has the form: 

\begin{equation}
\sigma_{t}\left(\bar{\epsilon}^{ck}\right)=\sigma_{t}^{iy}e^{\lambda\bar{\epsilon}^{ck}}\label{eqn:param2}
\end{equation}


Fig 4 Tensile hardening rule for the CDM model using two parameters
for an exponential approximation. In addition to the hardening rules,
the damage evolution equations must also be parameterized. 

The compressive damage, Dc, is assumed to be a linear function of
the inelastic strain through a compressive damage rate parameter,
m: 

\begin{equation}
D_{c}\left(\bar{\epsilon}^{in}\right)=\bar{\epsilon}^{in}m\label{eqn:param3}
\end{equation}


The tensile damage ($D_{t}$) evolution is slightly less trivial,
but can also be characterized by a single parameter due to some constraints
imposed on the function by the nature of the damage parameter. In
tension, the damage evolution curve starts at the origin and asymptotically
approaches $D_{t}=1$ as $\bar{\epsilon}^{ck}\rightarrow\infty$.
As such, under this functional assumption, the only parameter required
to describe this relationship is the tensile damage rate parameter,
n: 

\begin{equation}
D_{t}\left(\bar{\epsilon}^{ck}\right)=1-\frac{1}{\left(1+\bar{\epsilon}^{ck}\right)^{n}}\label{eqn:param4}
\end{equation}


Sample damage evolution curves for both tension and compression are
illustrated in Fig 5, where one can see that the rate at which the
tensile damage evolves is far larger than the rate at which the compressive
damage evolves. The combination of the elastic parameters, the hardening
rule parameters, and the damage evolution parameters, yield a total
of nine parameters that must be identified by experiments or through
up-scaling to define the behavior of CDM model.

Fig 5 Damage evolution for both tension and compression for the CDM
model using only one rate parameter.


\subsubsection{Drucker-Prager Plasticity with Ductile Damage}

Elastic parameters are same as above. Young's Modulus and Poisson's
Ratio - 2 elastic parameters

The Drucker-Prager plasticity model can be written in terms of 4 individual
components: The friction angle, dilation angle, initial tensile strength,
and the hardnening rule. The hardening rule assumes a functional form
in terms of a power law:

\begin{equation}
\sigma_{y}=A+B\left(\bar{\epsilon}^{pl}\right)^{n}\label{eqn:param5}
\end{equation}


This yeilds a total of 6 parameters for the plasticity characterization.
The damage charachterization is comprised of two main components,
the damage initiation and the damage evolution. The damage initiation
is based on the Johnson-Cook model as follows: 

\begin{equation}
\bar{\epsilon}_{D}^{pl}=D_{1}+D_{2}e^{D_{3}\eta}\label{eqn:param6}
\end{equation}


