\subsection{Damage-Plasticty Model For Quasi-Brittle Materials}

This damage-plasticity model was developed by \citet{lubliner_plastic-damage_1989}
as a plasticity based damage model for non-linear analysis of concrete
failure. Subsequently, \citet{lee_plastic-damage_1998} further developed the
model to facilitate cyclic loading by adding a second damage variable
and introducing a new yield function to account for the additional
damage variable. 

This model was specifically formulated for modeling quasi-brittle
materials under low confining stresses subject to cyclic loading.
In addition to the separate damage variables governing the stiffness
degradation, the stiffness recovery and material hardening/softening
is also treated separately in both compression and tension. Because
the formulation does not consider the effects of large hydrostatic
stresses, the applicability of this plastic to in-situ geomechanics
at depth may not be sufficiently accurate. As such, this model is
more appropriate for shallow geological models that require cyclic
loading paths to be considered. 

The yield function for this model is:

\begin{equation}
F\left(\boldsymbol{\bar{\sigma}},\boldsymbol{\bar{\epsilon}^{pl}}\right)=\frac{1}{1-\alpha}\left(\bar{q}-3\alpha\bar{p}+\beta\bar{\epsilon}^{pl}\left\langle \hat{\boldsymbol{\bar{\sigma}}}_{max}\right\rangle -\gamma\left\langle -\hat{\boldsymbol{\bar{\sigma}}}_{max}\right\rangle \right)-\bar{\boldsymbol{\sigma}}_{c}\left(\boldsymbol{\bar{\epsilon}_{c}^{pl}}\right)
\label{eqn:const10}
\end{equation}

In this formulation of damage-plasticity, the brittle nature of rock
necessitates separate characterization of tensile and compressive
damage. In the case where a rock sample fails completely in tension,
(i.e. the tensile stiffness becomes effectively 0), the compressive
strength can remain intact to a fairly high degree such that two separate
scalar damage values can exist for the given system. As such, the
equivalent plastic strain is also considered separately for tension
and compression and is represented as follows: 
\begin{equation}
\boldsymbol{\bar{\epsilon}^{pl}}=\begin{bmatrix}\boldsymbol{\bar{\epsilon}_{t}^{pl}}\\
\boldsymbol{\bar{\epsilon}_{c}^{pl}}
\end{bmatrix}\label{eqn:const9}
\end{equation}


hardening rule for this model:

\begin{equation}
\mathbf{h}\left(\boldsymbol{\bar{\sigma}},\boldsymbol{\bar{\epsilon}^{pl}}\right)=\left[\begin{array}{ccc}
r\left(\boldsymbol{\hat{\bar{\sigma}}}\right) & 0 & 0\\
0 & 0 & -\left(r\left(\boldsymbol{\hat{\bar{\sigma}}}\right)-1\right)
\end{array}\right]\label{eqn:const9-1}
\end{equation}

where:

\begin{equation}
r\left(\hat{\boldsymbol{\bar{\sigma}}}\right)=\frac{\sum_{i=1}^{3}\left\langle \hat{\boldsymbol{\bar{\sigma}}}_{i}\right\rangle }{\sum_{i=1}^{3}\left|\hat{\boldsymbol{\bar{\sigma}}}_{i}\right|},\qquad0\leq r\left(\hat{\boldsymbol{\bar{\sigma}}}\right)\leq1\label{eqn:const9-2}
\end{equation}


and $\left\langle \cdotp\right\rangle $ are Macauley brackets as
such....

\begin{equation}
\left\langle x\right\rangle =\frac{1}{2}\left(\left|x\right|+x\right)\label{eqn:const9-3}
\end{equation}


flow rule for this model from drucker-prager:

\begin{equation}
G\left(\boldsymbol{\bar{\sigma}}\right)=\sqrt{\left(\varepsilon\sigma_{t0}\tan\psi\right)^{2}-\bar{q}^{2}}-\bar{p}\tan\psi\label{eqn:const11}
\end{equation}


{*}{*}{*}{*}{*}Damage Evolution and Stiffness Degredation

The evolution of the equivalent plastic strains are formulated by
assuming the stress-strain curves can be converted into stress vs
plastic strain curves where the tensile and compressive stresses are
treated seperately: 
\begin{equation}
\begin{array}{c}
\boldsymbol{\sigma}_{t}=\boldsymbol{\sigma}_{t}\left(\boldsymbol{\bar{\epsilon}_{t}^{pl}},\boldsymbol{\dot{\bar{\epsilon}}_{t}^{pl}}\right)\\
\boldsymbol{\sigma}_{c}=\boldsymbol{\sigma}_{c}\left(\boldsymbol{\bar{\epsilon}_{c}^{pl}},\boldsymbol{\dot{\bar{\epsilon}}_{c}^{pl}}\right)
\end{array}
\label{eqn:dam1}
\end{equation}


Loading a quasi-brittle in compression or tension causes damage in
the material, which reduces the effective stiffness, weakening the
unloading response. This damage is characterized by two damage variables,
one of which represents the damage due to tensile loading, the other
represents damage due to compressive loading. 
\begin{equation}
\begin{array}{c}
D_{t}=D_{t}\left(\boldsymbol{\bar{\epsilon}_{t}^{pl}}\right),\qquad0\leq D_{t}\leq1\\
D_{c}=D_{c}\left(\boldsymbol{\bar{\epsilon}_{c}^{pl}}\right),\qquad0\leq D_{c}\leq1\end{array}
\label{eqn:dam2}
\end{equation}


The damage in both compression and tension is a neccesarily increasing
function of the equivalent plastic strains. This formulation will
adopt the convention where $\boldsymbol{\mathbf{\sigma}_{c}}$ is
positive in compression, as with the respective strains. 
\begin{equation}
\begin{array}{c}
\boldsymbol{\sigma}_{t}=(1-D_{t})\mathbf{E}:(\boldsymbol{\epsilon_{t}}-\boldsymbol{\bar{\epsilon}_{t}^{pl}})\\
\boldsymbol{\sigma}_{c}=(1-D_{c})\mathbf{E}:(\boldsymbol{\epsilon_{c}}-\boldsymbol{\bar{\epsilon}_{c}^{pl}})
\end{array}
\label{eqn:dam3}
\end{equation}


For cyclic loading, both the compressive and tensile damage need to
be considered. Two stiffness recovery factors are introduced, $s_{t}$
and $s_{c}$, which represent the stiffness recovery effects associated
with stress reversals. The damage can be said to take the form of:
\begin{equation}
(1-D)=(1-s_{t}D_{c})(1-s_{c}D_{t}),\qquad0\leq s_{t},s_{c},\leq1\label{eqn:dam4}
\end{equation}


In the case of tensile loading followed by compressive loading, the
