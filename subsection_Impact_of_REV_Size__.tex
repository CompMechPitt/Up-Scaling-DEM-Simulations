\subsection{Impact of REV Size on Estimated Parameters}

Due to the uncertainty of choosing an REV, the assumption of the REV size was tested using eight different sample REV radii and running the homogenization and parameter estimation algorithms for each. The assumed REV radius for the parameter estimation simulations was $4m$, which corresponds to an area of $50.24 m^2$. To test this assumption, the REV radii was sampled at $0.5m$ intervals to see where the resultant parameters converged.

Three of the 11 parameters of the constitutive model are shown in Figure \ref{fig:revconverge} as the REV size is increased. It can be seen in these plots that the material parameters converge at different sizes. This convergence variation illustrates part of the issue in defining an REV such that some parameters require a larger REV than others. That being said, for this granite rock mass, an REV of radius $3m$ or area of $28.26 m^2$ was deemed to be the minimum size based on the convergence of the last parameter to converge (the dilation angle in this case). Since the assumed REV size is larger than the minimum REV size, one is able to confirm the validity of the REV size assumption.