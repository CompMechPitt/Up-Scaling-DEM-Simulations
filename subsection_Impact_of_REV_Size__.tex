\subsection{Impact of REV Size on Estimated Parameters}

Due to the uncertainty of choosing an REV, the assumption of the REV size was tested using eight different sample REV radii and running the homogenization and parameter estimation algorithms for each. The assumed REV radius for the parameter estimation simulations was $4m$, which corresponds to an area of $50.24 m^2$. To test this assumption, the REV radii was sampled at $0.5m$ intervals to see where the resultant parameters converged.

The convergence of three of the 11 parameters is shown in Figure \ref{fig:revconverge} as a function of REV size. It can be seen in these plots that the material parameters converge at different sizes. This convergence variation illustrates part of the challenge in defining an REV because some parameters require a larger REV than others and it is not obvious a prior which parameters will dominate. That being said, for the granite rock mass considered, an REV of radius $3m$ or with an area of $28.26 m^2$ was deemed to be the minimum size based on the convergence of the dilation angle - the last parameter to converge. Since the assumed REV size is larger than the minimum REV size, one is able to confirm the validity of the REV size used here.