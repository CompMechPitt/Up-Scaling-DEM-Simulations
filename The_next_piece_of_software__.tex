The next piece of software required for up-scaling is the homogenization program, which reads the data exported from the DEM simulations in order to asses the homogenized stress and strain tensors. This homogenization program was built in-house in order to implement the homogenization algorithms described in this paper. This program was wrapped to yield output data files in a format that is conducive for the parameter estimation software to read. 

The parameter estimation software is the heart of the up-scaling framework. It is this component that drives the continuum FEM simulations to converge on the optimal parameter set. In this article, the model-independent parameter estimation and optimization software, OSTRICH\textsuperscript{TM} was chosen for it's versatility and expandability. This software is directed on how to produce input files for the FEM program to run. Once an optimal parameter set is converged upon, the software produces an output file containing the parameter set as well as a standard statistical analysis.

The FEM program is where the CDM models are solved. Here, the popular FEM software package ABAQUS\textsuperscript{TM} by SIMULIA\textsuperscript{TM} was chosen as it has a number of predefined CDM models which were found to represent geomaterials adequately. Again, a Python\textsuperscript{TM} wrapper script had to be written in order to create files for the parameter estimation software to read. 