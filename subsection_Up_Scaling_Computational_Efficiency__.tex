\subsection{Up-Scaling Computational Efficiency}

The CDM model for this case requires about two orders of magnitude less computational effort than the DEM model (Table \ref{tab:computation}). The CDM simulation uses a comparable number of continuum elements ($29,866$) as in the DEM simulation ($25,898$) for comparison and adequate convergence. The CDM model efficiency can be improved by applying a Selectively Refined Mesh (SRM) where only the areas with stress concentrations and large stress gradients have a strongly refined mesh. With the SRM, a converged CDM solution is achievable with only $3,577$ elements leading to another order of magnitude reduction in computational effort. 

\begin{table}[!htbp]
\centering
\caption{Comparison of Computational Time for the DNS}
\label{tab:computation}
\begin{tabular}{@{}ccccc@{}}
\toprule
\textbf{Simulation} & \textbf{No. Continuum} & \textbf{Processor} & \textbf{Slope Failure} & \textbf{Computational} \\ 
\textbf{Type} & \textbf{Elements} & \textbf{Clock Speed} & \textbf{Load} & \textbf{Time} \\ \midrule
DEM                      & $25,898$                         & $2.20 GHz$                    & $11.2 MPa$                  & $46.5 hr$                  \\
CDM                      & $29,866$                         & $1.80 GHz$                    & $11.5 MPa$                  & $0.65 hr$                  \\
CDM - SRM                      & $3,577$                         & $1.80 GHz$                    & $11.5 MPa$                  & $0.013 hr$                  \\ \bottomrule
\end{tabular}
\end{table}

The DEM simulation was run serially on a $2.2GHz$ CPU while the CDM simulation was run serially on a $1.8GHz$ CPU. Despite the CDM model having more continuum elements than the DEM model, and the CDM model running on a slower CPU, a decrease in computational time of the DEM simulation from $46.5 hr$ to $0.65 hr$ was observed. Running the CDM model with a SRM reduces the total computational time to $0.013 hr$, or eight minutes instead of two days. This large increase in computational efficiency with marginal decrease in model accuracy can be immensely useful for large scale geomechanical problems in NFR. 
