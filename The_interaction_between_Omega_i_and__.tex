
The interaction between $\Omega_i$ and $\Omega_j$ along $\Gamma_{ij}$ is the distinguishing feature in the DEM formulation, and is comprised of two main components: contact detection and the constitutive relationship. The contact detection algorithms are responsible for ensuring that $\Omega_i$ and $\Omega_j$ do not penetrate each other and ensuring that the appropriate contact forces are transferred between elements. These contact forces are governed by constitutive models of $\Gamma_{ij}$ which can be described in general by a shear stiffness, $k_s$, in a direction parallel to the $\Gamma_{ij}$, and a normal stiffness, $k_n$, in a direction normal to $\Gamma_i$. The normal stress in the discontinuity, $\sigma_n$, can be expressed as a function of the normal elastic displacement, $u_n$, up until the tensile strength, $T$, is exceeded: 

\begin{equation}
\sigma^n_i=\left\{\begin{matrix}
\sigma^n_i\left(k^n, u_i^n\right) &if& \sigma^n_i \geq -T\\ 
 0 & if &\sigma^n_i < -T
\end{matrix}\right.
\label{eqn:demnormal}
\end{equation}

Futhermore, the shear stress, $\tau_i$, in $\Gamma_{ij}$ can be written in terms of the elastic shear displacement, $u_i^s$, until the maximum shear strength, $\tau^{max}$ is reached. The point when the shear stress within $\Gamma_{ij}$ exceeds the prescribed maximum shear stress, the discontinuity experiences plastic shear displacements in order not to exceed the maximum shear stress:

\begin{equation}
\tau_i=\left\{\begin{matrix}
\tau_i\left(k^s,u_i^s, \sigma_i^n\right) &if&\left |\tau_i \right | < \tau^{max}\\ 
\frac{u_i^s}{\left|u_i_^s\right|}\tau^{max} & if &\left |\tau_i \right | \geq \tau^{max}
\end{matrix}\right.
\label{eqn:demshear}
\end{equation}

The stress fields within the elements are described by the internal motion of the element by standard continuum constitutive relationships in terms of the elastic stiffness tensor, $\mathbf{E}^m$, and the plastic strain, $\boldsymbol{\epsilon}^m_{pl}$:  

\begin{equation}
\boldsymbol{\sigma}^m =\mathbf{E}^m:\left(\boldsymbol{\epsilon}^m - \boldsymbol{\epsilon}^m_{pl}\right)
\label{eqn:demcont}
\end{equation}

With the constitutive behaviour of the discontinuities and the constitutive behaviour of the continuum blocks, the overall behaviour of the rock mass can be characterized through material properties of the rock discontinuities and the intact rock.