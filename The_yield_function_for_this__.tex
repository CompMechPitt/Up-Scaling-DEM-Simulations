The yield function for this model is based on the yield function proposed by \citet{lee_plastic-damage_1998} which was developed to allow for differential hardening under tension and compression. The resultant yield function, $F$, can be expressed as a function of effective stress and equivalent plastic strain:
\begin{equation}
    F
    \left(\bar{\sigma}_{ij},\bar{\epsilon}^{pl}\right)
    =
    \frac{1}{1-\alpha}
    \left(
        q\left(\bar{\sigma}_{ij}}\right)-3\alpha p\left(\bar{\sigma}_{ij}}\right)+\beta\left(\bar{\epsilon}^{pl}\right)
        \left\langle\hat{\bar{\sigma}}_{ij}\right\rangle-\gamma\left\langle-\hat{\bar{\sigma}}_{ij}\right\rangle\right)
    -\bar{\sigma}_{c}
    \left(
        \bar{\epsilon_{c}}^{pl}
    \right)
\label{eqn:const10}
\end{equation}

Where $\alpha$ and $\gamma$ are dimensionless material constants. Experimental testing has yielded values of $\alpha$ between 0.08 and 0.12, as well as a typical $\gamma$ value of approximately 3 \cite{lubliner_plastic-damage_1989}. 

In this formulation of damage-plasticity, the brittle nature of rock necessitates separate characterization of tensile and compressive damage. With quasi-brittle materials such as rock, it has been found that compressive stiffness can be recovered upon crack closure. Conversely, in these materials, tensile stiffness is not recovered after compressive cracks have developed. This behaviour implies that two separate scalar damage values should exist for the given system to account for both the compressive stiffness degradation and the tensile stiffness degradation. 

As such, the equivalent plastic strain is also considered separately for tension ($\bar{\epsilon}_{t}^{pl}$) and compression ($\bar{\epsilon}_{c}^{pl}$) and is represented as follows: 

\begin{equation}
\boldsymbol{\bar{\epsilon}^{pl}}=
\left[
\begin{array}{c}
    \bar{\epsilon}_{t}^{pl}\\
    \bar{\epsilon}_{c}^{pl}
\end{array}
\right]
\label{eqn:const9}
\end{equation}


The hardening rule for this model is slightly modified from equation \ref{eqn:const8d} to accommodate two hardening variables (equivalent plastic strains) for tension and compression. The hardening rule can thus be written in matrix form:

\begin{equation}
\mathbf{h}\left(\bar{\sigma}_{ij},\bar{\epsilon}^{pl}\right)=\left[\begin{array}{ccc}
r\left(\hat{\bar{\sigma}}_{ij}\right)\frac{\sigma_t\left(\bar{\epsilon}_{t}^{pl}\right)}{g_t} & 0 & 0\\
0 & 0 & -\left(r\left(\hat{\bar{\sigma}}_{ij}\right)-1\right)\frac{\sigma_c\left(\bar{\epsilon}_{c}^{pl}\right)}{g_c}
\end{array}\right]\label{eqn:const9-1}
\end{equation}

Where $\sigma_t$ and $\sigma_c$ are the yield stresses in tension and compression as specified by the hardening curves which describe the evolution of the equivalent plastic strains: 
\begin{equation}
\begin{array}{c}
\sigma_{t}=\sigma_{t}\left(\bar{\epsilon}_{t}^{pl},\dot{\bar{\epsilon}}_{t}^{pl}\right)\\
\sigma_{c}=\sigma_{c}\left(\bar{\epsilon}_{c}^{pl},\dot{\bar{\epsilon}}_{c}^{pl}\right)
\end{array}
\label{eqn:dam1}
\end{equation}

In addition, $g_t$ and $g_c$ represent the dissipated fracture energy density during micro-cracking. Furthermore, the weighting function, $r\left(\hat{\bar{\sigma}}\right)$, weights the hardening functions depending on the degree of tension or compression that the model is experiencing:

\begin{equation}
r\left(\hat{\boldsymbol{\bar{\sigma}}}\right)=\frac{\sum_{i=1}^{3}\left\langle \hat{\boldsymbol{\bar{\sigma}}}_{i}\right\rangle }{\sum_{i=1}^{3}\left|\hat{\boldsymbol{\bar{\sigma}}}_{i}\right|},\qquad0\leq r\left(\hat{\boldsymbol{\bar{\sigma}}}\right)\leq1\label{eqn:const9-2}
\end{equation}


Loading a quasi-brittle in compression or tension causes damage in
the material, which reduces the effective stiffness, weakening the
unloading response. This damage is characterized by two damage variables,
one of which represents the damage due to tensile loading, the other
represents damage due to compressive loading. 
\begin{equation}
\begin{array}{c}
D_{t}=D_{t}\left(\boldsymbol{\bar{\epsilon}_{t}^{pl}}\right),\qquad0\leq D_{t}\leq1\\
D_{c}=D_{c}\left(\boldsymbol{\bar{\epsilon}_{c}^{pl}}\right),\qquad0\leq D_{c}\leq1\end{array}
\label{eqn:dam2}
\end{equation}


The damage in both compression and tension is a necessarily increasing
function of the equivalent plastic strains. This formulation will
adopt the convention where $\boldsymbol{\mathbf{\sigma}_{c}}$ is
positive in compression, as with the respective strains. 
\begin{equation}
\begin{array}{c}
\boldsymbol{\sigma}_{t}=(1-D_{t})\mathbf{E}:(\boldsymbol{\epsilon_{t}}-\boldsymbol{\bar{\epsilon}_{t}^{pl}})\\
\boldsymbol{\sigma}_{c}=(1-D_{c})\mathbf{E}:(\boldsymbol{\epsilon_{c}}-\boldsymbol{\bar{\epsilon}_{c}^{pl}})
\end{array}
\label{eqn:dam3}
\end{equation}

For cyclic loading, both the compressive and tensile damage need to
be considered. Two stiffness recovery factors are introduced, $s_{t}$
and $s_{c}$, which represent the stiffness recovery effects associated
with stress reversals. The damage can be said to take the form of:
\begin{equation}
(1-D)=(1-s_{t}D_{c})(1-s_{c}D_{t}),\qquad0\leq s_{t},s_{c},\leq1\label{eqn:dam4}
\end{equation}

etc...
