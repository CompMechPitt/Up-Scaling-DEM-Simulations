However, one must also note the potential displacement jumps that
may occur between blocks on the boundary as the model deforms. In
the case where the blocks become physically separated, there exists
a discontinuity along the homogenization boundary as can be seen in figure
\ref{fig:homoboundary}. These discontinuities along the homogenization
boundary were considered by adding boundary segments to the homogenization
boundary between the corners of the adjacent blocks. As such, steps
\ref{hli:9} and \ref{hli:10} find coincident corners on adjacent
boundary blocks to allow for the blocks to become separated on the
boundary, while still maintaining connectivity along the homogenization
domain boundary. It is also necessary to order the corners (step \ref{hli:11})
as they would appear along the homogenization boundary in order to
define the boundary segments along which integration can be performed.

