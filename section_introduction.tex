\section{Introduction}
Discrete Element Method (DEM) models are used commonly in geomechanics to explicitly model the mechanics of Naturally Fractured Rock (NFR) masses (Jing, 2003). NFR is often modeled as a multiscale material due to the vastly different length scales involved in the deformation process (Zhou et al., 2003). At the fracture scale (10-3 m), the physics is dominated by brittle fracture propagation and fracture-to-fracture contact force interaction, while one is normally interested in the reservoir scale (103 m) response as a result of the spatial extension of the fractures. Because these scales of interest span approximately six orders of magnitude, multiscale methods are required to assess the overall response as modelling with fracture scale resolution at the reservoir scale becomes computationally prohibitive. 

DEM models, unlike standard continuum models, consider the fractures within the rock mass as a Discrete Fracture Network (DFN), which explicitly defines the geometry of the fracture network. The physics of block interaction is then governed by the motion, contact forces and traction-separation laws between the rock blocks and the fractures \cite{Cundall_1979}. Because NFR behavior is complex, even sophisticated phenomenological constitutive relationships may be inadequate to describe the complete rock mass behavior. The DEM approach aims to address this continuum behavioral deficiency by only requiring constitutive relations for the block interactions \cite{Cundall_2001}.

That being said, the main issue with DEM models is primarily the computational demands. Due to the large number of degrees of freedom in the models and the requirement for very small time steps —because of the constant need for contact detection between blocks — running reservoir scale models is computationally prohibitive. The intent of this article is to develop a framework that incorporates the response of the DEM models while harnessing the computational speed of the continuum models. Up-scaling is accomplished in this paper by ‘calibrating’ a continuum model with DEM virtual experimental data using an iterative least squares regression algorithm.

The general goal of up-scaling is to formulate simplified coarse-scale governing equations that approximate the fine-scale behavior of a material \cite{Geers_2010}. In the case of the DEM simulations in this investigation, the aim of up-scaling is to identify the parameters of a continuum model that best mimics the response of the DEM model.

Multiscale methods that can be considered often fall into one of two classes: hierarchical or concurrent \cite{Gracie_2011}. In concurrent multiscale models, different scales are used in different regions of the domain; the solution of the coupled model proceeds by solving both scales simultaneously. This approach is very expensive since the time step of the whole simulation is controlled by the fine-scale model; however, the solution is often more accurate. In hierarchical multiscale methods, the constitutive behavior at the coarser scale is determined by exercising a finer scale RVE. The finer scale models vary from relatively simple models, as in micromechanics, to complex nonlinear models. This approach is much more efficient, but can be less accurate. Up-scaling in this investigation can be considered to be a hierarchical multiscale method using computational homogenization. 

Many multiscale homogenization techniques have been developed and proposed in the past, \cite{Aanonsen_2006,Temizer_2009,Loehnert_2005}, but none have addressed the problem of up-scaling DEM simulations of  NFR to continuum damage mechanics models using parameter estimation techniques.
