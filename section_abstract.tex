\section{Abstract}
Discrete Element Methods (DEM) explicitly and accurately model the mechanics of naturally occurring fractured rock masses. However, due to the large number of degrees of freedom in DEM simulations and the requirement of small times steps, the application of DEM simulations to reservoir scale problems and long-term fluid injection is computationally prohibitive. 

In order to reduce the computational costs associated with full-scale DEM simulations, an up-scaling method is presented in which Representative Elementary Volume (REV) DEM simulations are used to calibrate the parameters of a Continuum Damage Mechanics (CDM) constitutive model. The CDM model empirically captures the effect of the degradation of the rock integrity due to the yielding and sliding of natural fractures in the rock mass. 

Up-scaling is achieved through homogenization, in which the spatially averaged stress-strain behaviour of various DEM RVE simulations is computed. Subsequently, a CDM constitutive relationship fitted using Levenberg-Marquardt Algorithm (LMA) and the homogenized DEM simulation data. The CDM model is then used in reservoir scale simulations. The CDM model is implemented in ABAQUS\textsuperscript{TM} and DEM simulations were conducted using UDEC\textsuperscript{TM}. The up-scaling methodology is demonstrated through a case study on a naturally fractured carbonate reservoir; the up-scaled CDM model is compared to direction numerical simulation with the DEM model.
